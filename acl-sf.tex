\documentclass[aspectratio=169]{beamer}
\usepackage[utf8]{inputenc}
\usepackage[T1]{fontenc}

\usetheme{focus}
\definecolor{main}{RGB}{11, 93, 25}

\usecolortheme[RGB={11,93,20}]{structure}
\usepackage[spanish]{babel}

\usepackage{colortbl}
\usepackage{color}
\usepackage{pifont}
\usepackage{ulem}



\usepackage{colortbl}
\usepackage{color}
\usepackage{pifont}
%\usepackage[normalem]{ulem}                                                                 \
                                                                                              
\usepackage{listings}

\parskip=12pt

\lstset{basicstyle=\normalsize,
aboveskip=5pt,
basicstyle=\small\ttfamily,
belowskip=5pt
}

\author{Alberto Molina Coballes}
\title{Sistemas de ficheros: Listas de control de acceso}
\institute{IES Gonzalo Nazareno}
\titlegraphic{\includegraphics[width=1.5cm]{cc_by_sa.png}}
\logo{\includegraphics[width=.75cm]{logo_iesgn.png}}
\date{\today}

\definecolor{verde}{rgb}{0,0.73,0}

\begin{document}

\def\braces#1{[#1]}

\begin{frame}[t,plain]
\titlepage
\end{frame}

\begin{frame}[fragile]
  \frametitle{acl sobre los ficheros}
  \begin{itemize}
  \item Permite un control adicional sobre los permisos de los ficheros
  \item Debe estar habilitado en el sistema de ficheros
  \item Se definen con \texttt{setfacl} del paquete \texttt{acl}
  \item Ejemplo:
\footnotesize{
\begin{verbatim}
setfacl -m ``u:usuario:rw'' fichero      
\end{verbatim}}
  \item Se comprueban las ACLs con \texttt{getfacl}
  \item Cuando hay ACLs definidas se indica con un \texttt{+} al listar:
\footnotesize{
\begin{verbatim}
drwxrwxr-x+  2 alberto alberto        6 ene 19 12:15 Directorio
    \end{verbatim}}
  \item Cuidado con el uso de ACLs o esto va a parecerse a Windows ;)
  \end{itemize}
\end{frame}
\end{document}
