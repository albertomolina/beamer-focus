\documentclass[aspectratio=169]{beamer}
\usepackage[utf8]{inputenc}
\usepackage[T1]{fontenc}

\usetheme{focus}
\definecolor{main}{RGB}{11, 93, 25}

\usecolortheme[RGB={11,93,20}]{structure}
\usepackage[spanish]{babel}

\usepackage{colortbl}
\usepackage{color}
\usepackage{pifont}
\usepackage{ulem}



\usepackage{colortbl}
\usepackage{color}
\usepackage{pifont}
%\usepackage[normalem]{ulem}
\usepackage{listings}

\parskip=12pt

\lstset{basicstyle=\normalsize,
aboveskip=5pt,
basicstyle=\tiny\ttfamily,
belowskip=5pt
}

\author{Alberto Molina Coballes}
\title{Sistemas de Ficheros en Linux}
\institute{IES Gonzalo Nazareno}
\titlegraphic{\includegraphics[width=1.5cm]{cc_by_sa.png}}
\logo{\includegraphics[width=.75cm]{logo_iesgn.png}}
\date{\today}

\definecolor{verde}{rgb}{0,0.73,0}

\begin{document}

\def\braces#1{[#1]}

\begin{frame}[t,plain]
\titlepage
\end{frame}

\begin{frame}[fragile]
  \frametitle{Fichero}
  \begin{itemize}
  \item Es un conjunto de datos almacenados en un dispositivo de
    almacenamiento
  \item Un fichero posee un nombre y metadatos
  \item El tipo de datos contenidos está definido por su formato y en
    ocasiones por su extensión
  \item Normalmente el formato está definido en la cabecera
    (\textit{magic number})
  \item Un fichero solo con datos ASCII o UTF8 se conoce como texto
    plano (\textit{plain text})
  \item Los ficheros con datos reciben el nombre de ficheros regulares
  \end{itemize}
  \begin{lstlisting}
    file fichero.jpg
    xxd fichero.pdf |head
\end{lstlisting}
\small{\url{https://asecuritysite.com//forensics/magic}}
\end{frame}

\begin{frame}
  \frametitle{Ficheros especiales}
  Aparte de los ficheros regulares, en UNIX ``todo es un fichero'':
  \begin{itemize}
  \item Enlaces
  \item sockets
  \item Dispositivos (En \texttt{/dev})
  \item Ficheros virtuales (En \texttt{/proc} o \texttt{/sys})
  \item Ficheros virtuales en memoria (Tipo tmpfs o ramfs)
  \end{itemize}
\end{frame}

\begin{frame}
  \frametitle{Directorios}
  \begin{itemize}
  \item Los directorios son contenedores de ficheros y se pueden
    anidar (subdirectorios)
  \item El directorio principal es el directorio raíz o /
  \item Los directorios se organizan formando un árbol a partir del
    directorio raíz
  \item Los ficheros se definen de forma única por el nombre que
    incluye su ruta completa desde el directorio raíz,
    p.ej. \texttt{/usr/share/doc/apt/copyright}
  \item La estructura de los directorios es estricta y definida en el
    Linux Filesystem Hierarchy Standard (FHS) de la Linux Foundation
  \item Las diferentes distros de GNU/Linux deben seguir la FHS
  \end{itemize}
\small{\url{https://refspecs.linuxfoundation.org/fhs.shtml}}
\end{frame}

\begin{frame}[fragile]
  \frametitle{Ejemplos de ficheros}
  \begin{lstlisting}
stat sist-ficheros.pdf 
  File: sist-ficheros.pdf
  Size: 135886    	Blocks: 272        IO Block: 4096   regular file
Device: fd02h/64770d	Inode: 68450572    Links: 1
Access: (0644/-rw-r--r--)  Uid: ( 1000/ alberto)   Gid: ( 1000/ alberto)
Access: 2019-12-17 17:28:46.430874348 +0100
Modify: 2019-12-17 20:04:19.925574364 +0100
Change: 2019-12-17 20:04:19.925574364 +0100
 Birth: -

stat /etc/
  File: /etc/
  Size: 12288     	Blocks: 32         IO Block: 4096   directory
Device: fd00h/64768d	Inode: 33554560    Links: 167
Access: (0755/drwxr-xr-x)  Uid: (    0/    root)   Gid: (    0/    root)
Access: 2015-11-07 21:24:49.164146678 +0100
Modify: 2019-12-17 16:55:39.973704262 +0100
Change: 2019-12-17 16:55:39.973704262 +0100
 Birth: -
  \end{lstlisting}
\end{frame}

\begin{frame}
  \frametitle{Sistemas de ficheros}
  \begin{itemize}
  \item Controla cómo se almacenan y obtienen los datos
  \item Capa lógica
    \begin{itemize}
    \item Responsable de la interacción con los usuarios y programas
    \item API
    \item Modelo de seguridad
    \end{itemize}
  \item Implementación: Enlaza la capa lógica con los dispositivos de
    almacenamiento
  \end{itemize}
\end{frame}

\begin{frame}
  \frametitle{Dispositivos de bloques}
  \begin{itemize}
    \item Los dispositivos son ficheros especiales ubicados en
      \texttt{/dev}
    \item Dispositivos de caracteres (\textit{character devices})
    \item Dispositivos de bloques (\textit{block devices})
      \begin{itemize}
      \item E/S en bloques de datos
      \item Uso de buffer
      \item Usados para los dispositivos de almacenamiento
      \end{itemize}
    \item Discos duros, disquetes, CDs, DVDs, particiones, volúmenes
      lógicos, RAID, etc.
  \end{itemize}
\end{frame}

\begin{frame}
  \frametitle{Linux Virtual FileSystem (VFS)}
  \begin{itemize}
  \item Capa del kérnel que proporciona una interfaz única a usuarios
    y programas
  \item Implementa un único árbol aunque esté formado por varios
    dispositivos de bloques y/o tipos de sistemas de ficheros
  \item syscalls: open(2), stat(2), read(2), write(2) y chmod(2)
  \item Montar y desmontar
  \item Elementos principales de cada tipo de sistema de ficheros:
    \begin{itemize}
    \item Superbloque
    \item Inodos
    \end{itemize}
  \end{itemize}
\end{frame}

\begin{frame}
  \frametitle{DAS, NAS y SAN}
  \begin{itemize}
  \item \textit{Direct Attached Storage} (DAS)
  \item \textit{Network Attached Storage} (NAS)
  \item \textit{Storage Area Network} (SAN)
  \end{itemize}
\end{frame}

\begin{frame}
  \frametitle{Sistemas de ficheros para DAS}
  \begin{itemize}
  \item ext2/3/4
  \item xfs
  \item jfs
  \item reiserfs
  \item vfat
  \item zfs
  \item btrfs
  \end{itemize}
\end{frame}

\begin{frame}
  \frametitle{ext4}
  \begin{itemize}
  \item Sistema de ficheros por defecto en linux
  \item Características principales:
    \begin{itemize}
    \item Compatible con ext2/3
    \item Utiliza \textit{journaling}
    \item Hasta 1 EiB con ficheros de 16 TiB
    \item Mejoras en el rendimiento frente a ext2/3
    \item Herramientas del espacio de usuario: \texttt{e2fsprogs}:
      \texttt{mkfs.ext4 fsck.ext4 tune2fs resize2fs dumpe2fs debugfs ...}
    \end{itemize}
  \item No supone una mejora radical frente a ext3
  \end{itemize}
  \small{\url{https://www.slashroot.in/understanding-file-system-superblock-linux}}
\end{frame}

\begin{frame}
  \frametitle{xfs}
  \begin{itemize}
  \item Desarrollado originalmente por SGI
  \item Utilizado por defecto en RHEL 7
  \item Hasta 8 EiB
  \item Utiliza \textit{journaling}
  \item Permite redimensionado en ``caliente''
  \item Presume de tener mejor rendimiento que ext4
  \item Herramientas del espacio de usuario: \texttt{xfsprogs}:
     \texttt{mkfs.xfs fsck.xfs xfs\_db xfs\_growfs xfs\_info ...}
  \end{itemize}
  \small{\url{https://righteousit.wordpress.com/2018/05/21/xfs-part-1-superblock/}}
\end{frame}

\begin{frame}
  \frametitle{vfat}
  \begin{itemize}
  \item Sistema de ficheros de MS-DOS/Windows
  \item No guarda información de propietarios: sistema monousuario
  \item Enormes limitaciones
  \item Utilizado masivamente en dispositivos extraíbles
  \item VFS permite su utilización ``transparente'' en linux
  \item Herramientas del espacio de usuario: \texttt{dosfstools}:
    \texttt{mkfs.vfat fsck.vfat dosfslabel}
  \end{itemize}
\end{frame}
\end{document}
