\documentclass[aspectratio=169]{beamer}
\usepackage[utf8]{inputenc}
\usepackage[T1]{fontenc}

\usetheme{focus}
\definecolor{main}{RGB}{11, 93, 25}

\usecolortheme[RGB={11,93,20}]{structure}
\usepackage[spanish]{babel}

\usepackage{colortbl}
\usepackage{color}
\usepackage{pifont}
\usepackage{ulem}



\usepackage{colortbl}
\usepackage{color}
\usepackage{pifont}
%\usepackage[normalem]{ulem}
\usepackage{listings}

\parskip=12pt

\lstset{basicstyle=\normalsize,
aboveskip=5pt,
basicstyle=\small\ttfamily,
belowskip=5pt
}

\author{Alberto Molina Coballes}
\title{iSCSI}
\institute{IES Gonzalo Nazareno}
\titlegraphic{\includegraphics[width=1.5cm]{cc_by_sa.png}}
\logo{\includegraphics[width=.75cm]{logo_iesgn.png}}
\date{\today}

\definecolor{verde}{rgb}{0,0.73,0}

\begin{document}

\def\braces#1{[#1]}

\begin{frame}[t,plain]
\titlepage
\end{frame}

\begin{frame}
  \frametitle{iSCSI}
  \begin{itemize}
  \item Proporciona acceso a dispositivos de bloques sobre TCP/IP
  \item Se utiliza fundamentalmente en redes de almacenamiento
  \item Alternativa económica a Fibre Channel
  \item Utilizado típicamente en redes de cobre de 1 Gbps o 10 Gbps
  \end{itemize}
\end{frame}

\begin{frame}
  \frametitle{Elementos iSCSI}
  \begin{itemize}
  \item Unidad lógica (LUN): Dispositivo de bloques a compartir por el servidor iSCSI
  \item Target: Recurso a compartir desde el servidor. Un target incluye uno o varios LUN
  \item Initiator: Cliente iSCSI
  \item Multipath
  \item IQN es el formato más extendido para la descripción de los recursos. Ejemplo:

    \texttt{iqn.2020-01.es.tinaja:sdb4}
  \item iSNS: Protocolo que permite gestionar recursos iSCSI como si fueran Fibre Channel
  \end{itemize}
\end{frame}

\begin{frame}
  \frametitle{Implementaciones iSCSI}
  \begin{itemize}
  \item iSCSI tiene soporte en la mayoría de sistemas operativos
  \item En Linux usamos
    \href{https://github.com/open-iscsi/open-iscsi}{open-iscsi} como
    initiator
  \item Existen varias opciones en Linux para el servidor iSCSI:
    \begin{itemize}
    \item \href{http://linux-iscsi.org/wiki/Main_Page}{Linux-IO (LIO)}
    \item \href{http://stgt.sourceforge.net/}{tgt}
    \item \href{http://scst.sourceforge.net/}{scst}
    \item \href{http://www.peach.ne.jp/archives/istgt/}{istgt}
    \end{itemize}
  \end{itemize}
\end{frame}

\begin{frame}
  \frametitle{Demo}
  \small{\url{https://gist.github.com/albertomolina/6c621aee3f80c5e7baf3c111df670cf0}}
\end{frame}
\end{document}
