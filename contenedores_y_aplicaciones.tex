\documentclass[aspectratio=169]{beamer}
\usepackage[utf8]{inputenc}
\usepackage[T1]{fontenc}

\usetheme{focus}
\definecolor{main}{RGB}{11, 93, 25}

\usecolortheme[RGB={11,93,20}]{structure}
\usepackage[spanish]{babel}

\usepackage{colortbl}
\usepackage{color}
\usepackage{pifont}
\usepackage{ulem}



\usepackage{colortbl}
\usepackage{color}
\usepackage{pifont}
% \usepackage[normalem]{ulem}                                                                                                                                
\usepackage{hyperref}
\hypersetup{
    colorlinks=true,
    linkcolor=blue,
    filecolor=magenta,
    urlcolor=cyan,
}

\usepackage{listings}

\parskip=12pt

\lstset{basicstyle=\normalsize,
aboveskip=5pt,
basicstyle=\small\ttfamily,
belowskip=5pt
}

\author{Alberto Molina Coballes\\
José Domingo Muñoz Rodríguez}
\title{Contenedores y apliaciones}
\institute{IES Gonzalo Nazareno}
\titlegraphic{\includegraphics[width=1.5cm]{cc_by_sa.png}}
\logo{\includegraphics[width=.75cm]{logo_iesgn.png}}
\date{\today}

\definecolor{verde}{rgb}{0,0.73,0}

\begin{document}
\begin{frame}[t,plain]
\titlepage
\end{frame}

\begin{frame}
  \frametitle{Contenedores}
  \begin{itemize}
  \item Contenedores
  \item ¿Para qué sirve un sistema operativo?¿Qué es un proceso?
  \item Compartir o no compartir, ésa es la cuestión: .so., dependencias
  \item ¿Qué es un contenedor y para qué se utiliza?
  \item Precedentes en linux:
    \begin{itemize}
    \item chroot
    \item OpenVZ
    \item Linux vservers
    \end{itemize}
  \item Precedentes en otros sistemas operativos: FreeBSD Jails, Solaris Zones, etc.
  \end{itemize}
\end{frame}

\begin{frame}
  \frametitle{Contenedores}
  \begin{itemize}
  \item El gran hito: inclusión de cgroups y namespaces en el kérnel linux (a partir de 2007)
  \item cgroups (límite de memoria, cpu, I/O o red para un proceso y sus hijos)
    \url{https://wiki.archlinux.org/index.php/Cgroups}
  \item cgroupsv2 (rootless containers)
    \url{https://medium.com/nttlabs/cgroup-v2-596d035be4d7}
  \item namespaces: proporcionan un punto de vista diferente a un proceso (interfaces de red, procesos, usuarios, etc.)
    \url{http://laurel.datsi.fi.upm.es/~ssoo/SOA/namespaces.html}
  \item Todo esto unido a la expansión de linux en el centro de datos ha provocado la explosión en el uso de contenedores de los últimos años
  \end{itemize}
\end{frame}

\begin{frame}
  \frametitle{Despliegue de aplicaciones}
\end{frame}

\begin{frame}
  \frametitle{Aplicación monolítica}
  \begin{itemize}
  \item Todos los componentes en el mismo nodo
  \item Escalado vertical
  \item Arquitectura muy sencilla
  \item Suele utilizarse un solo lenguaje (o un conjunto pequeño de ellos)
  \item No pueden utilizarse diferentes versiones de un lenguaje a la vez
  \item Interferencias entre componentes en producción
  \item Complejidad en las actualizaciones. Puede ocasionar paradas en producción
  \item Normalmente usan infraestructura estática y fija por años
  \item Típicamente la aplicación no es tolerante a fallos
  \end{itemize}
\end{frame}

\begin{frame}
  \frametitle{Aplicación distribuida}
  \begin{itemize}
  \item Idealmente un componente por nodo
  \item Escalado horizontal
  \item Arquitectura más compleja
  \item Menos interferencias entre componentes
  \item Mayor simplicidad en las actualizaciones
  \item Diferentes enfoques no excluyentes: SOA, cloud native, microservicios, \ldots
  \end{itemize}
\end{frame}

\begin{frame}
  \frametitle{SOA}
  \begin{itemize}
  \item SOA: Service Oriented Architecture
  \item Servicios independientes
  \item Múltiples tecnologías, lenguajes y/o versiones interactuando
  \item Comunicación vía SOAP
  \item Uso de XML, XSD y WSDL
  \item Colas de mensajes
  \item Se relaciona con aplicaciones corporativas
  \item Se le achaca mucha complejidad y no ha terminado de extenderse
  \end{itemize}
\end{frame}

\begin{frame}
  \frametitle{Cloud Native Application}
  \begin{itemize}
  \item Énfasis en la adaptación de la infraestructura a la demanda
  \item Uso extensivo de la elasticidad: Infraestructura dinámica
  \item Aplicaciones resilientes
  \item Elasticidad horizontal
  \item Automatización
  \item Puede ser complejo de implementar en una aplicación
  \end{itemize}
\end{frame}

\begin{frame}
  \frametitle{Aplicación con o sin estado}
  \begin{columns}
    \column{0.5\textwidth}
    \begin{block}{Aplicaciones con estado}
      \begin{itemize}
      \item La operación se realiza en un contexto de un estado anterior
      \item Típicamente requerirá que se interactúe con el mismo servidor
      \item Suelen hacer uso de algún tipo de almacenamiento permanente
      \item Para escalar se pueden utilizar clusters de alta disponibilidad
      \end{itemize}
      Ejemplo: Webmail
    \end{block}
    \column{0.5\textwidth}
    \begin{block}{Aplicaciones sin estado}
      \begin{itemize}
      \item Cada operación independiente de las anteriores o posteriores
      \item Puede interactuar cada vez con un servidor diferente
      \item No necesita almacenamiento permanente o solo de lectura
      \item Para escalar puede utilizar clusters de balanceo de carga
      \end{itemize}
      Ejemplo: Buscador de Internet
    \end{block}
  \end{columns}
\end{frame}

\begin{frame}
  \frametitle{Microservicios}
  \begin{itemize}
  \item Deriva del esquema SOA
  \item No existe una definición formal, ni WSDL, XSD, etc. Solución más pragmática
  \item Servicios llevados a la mínima expresión (idealmente un proceso por nodo): Microservicios
  \item Comunicación vía HTTP REST y colas de mensajes, ¿gRPC?
  \item Relacionado con procesos ágiles de desarrollo, facilita enormemente la actualizaciones de
    versiones, llegando incluso a la entrega continua o despliegue continuo.
  \item Suele implementarse sobre contenedores
  \item Muy adecuada para aplicaciones sin estado
  \item Aumento de la latencia entre componentes
  \item Puede utilizar características de “cloud native application”
  \end{itemize}
\end{frame}

\begin{frame}
  \frametitle{Microservicios. Ejemplo}
  OpenStack
  \begin{itemize}
  \item Cada componente es un microservicio que puede ejecutarse en un nodo independiente
  \item kolla-ansible: Despliegue de OpenStack con ansible en múltiples contenedores docker
  \end{itemize}
  \url{https://elatov.github.io/2018/01/openstack-ansible-and-kolla-on-ubuntu-1604/}
\end{frame}
\end{document}
