\documentclass[aspectratio=169]{beamer}
\usepackage[utf8]{inputenc}
\usepackage[T1]{fontenc}

\usetheme{focus}
\definecolor{main}{RGB}{11, 93, 25}

\usecolortheme[RGB={11,93,20}]{structure}
\usepackage[spanish]{babel}

\usepackage{colortbl}
\usepackage{color}
\usepackage{pifont}
\usepackage{ulem}



\usepackage{colortbl}
\usepackage{color}
\usepackage{pifont}
%\usepackage[normalem]{ulem}
\usepackage{listings}

\parskip=12pt

\lstset{basicstyle=\normalsize,
aboveskip=5pt,
basicstyle=\small\ttfamily,
belowskip=5pt
}

\author{Alberto Molina Coballes}
\title{Btrfs}
\institute{IES Gonzalo Nazareno}
\titlegraphic{\includegraphics[width=1.5cm]{cc_by_sa.png}}
\logo{\includegraphics[width=.75cm]{logo_iesgn.png}}
\date{\today}

\definecolor{verde}{rgb}{0,0.73,0}

\begin{document}

\def\braces#1{[#1]}

\begin{frame}[t,plain]
\titlepage
\end{frame}

\begin{frame}
  \frametitle{El proyecto Btrfs}
  \begin{itemize}
  \item 2008: Comienza el desarrollo por empleados de Oracle
  \item 2009: Se incluye Btrfs en el kérnel linux
  \item 2012: Algunas distros comienzan a ofrecer Btrfs en sus instalaciones no experimentales
  \item 2015: Btrfs por defecto en Suse Linux Enterprise Server 12
  \item 2019: Red Hat publica RHEL 8 sin soporte para Btrfs
  \end{itemize}
\end{frame}

\begin{frame}
  \frametitle{Características de Btrfs}
  \begin{itemize}
  \item Es un sistema completo de almacenamiento que no requiere
    otras herramientas
  \item Licencia GPL y completamente integrado en el kérnel linux
  \item Hace años que se considera estable, aunque aún le falten
    algunas características de ZFS
  \item Gestiona los dispositivos de bloques directamente
  \item Incluye su propia implementación de RAID
  \item CoW, deduplicación, instantáneas, compresión, cifrado, \ldots
  \item Autoreparación
  \item No está extendido su uso como cabría esperar
  \end{itemize}
  \small{\url{https://btrfs.wiki.kernel.org/index.php/Status}}
\end{frame}

\begin{frame}
  \frametitle{Instalación en Debian}
  \begin{itemize}
  \item Instalación de las herramientas del espacio de usuario: \texttt{btrfs-progs}
  \end{itemize}
\end{frame}

\begin{frame}
  \frametitle{Demo}
\end{frame}
\end{document}
