\documentclass[aspectratio=169]{beamer}
\usepackage[utf8]{inputenc}
\usepackage[T1]{fontenc}

\usetheme{focus}
\definecolor{main}{RGB}{11, 93, 25}

\usecolortheme[RGB={11,93,20}]{structure}
\usepackage[spanish]{babel}

\usepackage{colortbl}
\usepackage{color}
\usepackage{pifont}
\usepackage{ulem}



\usepackage{colortbl}
\usepackage{color}
\usepackage{pifont}
\usepackage{soul}
%\usepackage[normalem]{ulem}

\usepackage{listings}

\parskip=12pt

\lstset{basicstyle=\normalsize,
aboveskip=5pt,
basicstyle=\small\ttfamily,
belowskip=5pt
}

\author{Alberto Molina Coballes}
\title{La Administración de Sistemas ha muerto ¡Viva la Administración de Sistemas!}
\institute{IES Gonzalo Nazareno}
\titlegraphic{\includegraphics[width=1.5cm]{cc_by_sa.png}}
\logo{\includegraphics[width=.75cm]{logo_iesgn.png}}
\date{29 de noviembre de 2019}

\definecolor{verde}{rgb}{0,0.73,0}

\begin{document}
\def\braces#1{[#1]}

\begin{frame}[t,plain]
\titlepage
\end{frame}

\begin{frame}
  \frametitle{Un día recibimos una llamada ...}
  \begin{center}
    \includegraphics[width=.6\textwidth]{img/Vintage-Telephone-Silhouette.png}
  \end{center}
\end{frame}

\begin{frame}
  \frametitle{¿Qué es un administrador de sistemas?}
  Se encarga de:
  \begin{itemize}
  \item Mantener las aplicaciones funcionando de forma segura
  \item Controlar los datos
  \end{itemize}
\end{frame}

\begin{frame}
  \frametitle{Roles}
  Las funciones se concretan en diferentes roles clásicos:
  \begin{itemize}
  \item Administrador Sistemas Operativos
  \item Administrador de Servicios: Correo, Web, etc.
  \item Gestor del Hardware
  \item Administrador de Redes
  \item Encargado de la Seguridad
  \item Administrador de Bases de datos
  \item Administrador del Almacenamiento
  \item Administrador de la Virtualización (hipervisores)
  \end{itemize}
  Roles mejor o peor cubiertos actualmente por la normativa de ASIR
\end{frame}

{
\usebackgroundtemplate{\centering{\includegraphics[width=.97\paperwidth]{img/800px-Cartoon_cloud.png}}}
\begin{frame}
  \frametitle{La nube}
  ``Como las aplicaciones se ejecutan en la nube de forma segura y los
  datos están en la nube, no necesitamos que nadie se encargue de esos
  datos ni de esas aplicaciones.''
\end{frame}
}

\begin{frame}
  \frametitle{¿Realmente todo va a estar en la nube?}
  Se estima que el 83\% de la carga de trabajo empresarial en el 2020 estará en la nube:
  \begin{itemize}
  \item 41\% en nube pública (AWS, GCP, Azure, etc.)
  \item 20\% en nube privada
  \item 22\% en nube híbrida
  \end{itemize}
  \href{https://www.forbes.com/sites/louiscolumbus/2018/01/07/83-of-enterprise-workloads-will-be-in-the-cloud-by-202\
0/}{Forbes: 83\% Of Enterprise Workloads Will Be In The Cloud By 2020}
\end{frame}

\begin{frame}
  \frametitle{¿En qué nube?}
  Dos situaciones muy diferentes:
  \begin{itemize}
  \item Una empresa decide dejar de gestionar su correo electrónico y
    contrata Google Suite (SaaS)
  \item Una empresa decide dejar de alojar localmente sus servidores
    de correo y mueve sus servidores de correo a AWS (IaaS)
  \end{itemize}
  La persona encargada del correo en cada situación tiene roles muy diferentes.
\end{frame}

\begin{frame}
  \frametitle{¿Ha muerto la Administración de Sistemas?}
  \begin{itemize}
  \item Se supone que determinadas tareas serán cada vez menos frecuentes:
    \begin{itemize}
    \item Mantenimiento de los sistemas operativos y aplicaciones locales
    \item Instalación y configuración de aplicaciones en redes locales
    \item Gestión local del almacenamiento
    \end{itemize}
  \item Ciertos roles se convertirán en nichos, no desaparecerán
  \item Siempre hará falta un Administrador de Sistemas clásico al que preguntar
  \end{itemize}
\end{frame}

\begin{frame}
  \frametitle{¡Viva la Administración de Sistemas!}
  \begin{itemize}
  \item No ha muerto, ha evolucionado
  \item Sería necesario que nos adaptásemos
  \item ¿Se va a adaptar la normativa?
  \item ¿Vamos a recibir formación?
  \end{itemize}
\end{frame}

\begin{frame}
  \frametitle{Nuevas oportunidades}
  \begin{itemize}
  \item Administración de la infraestructura en nube
  \item Arquitectura de las aplicaciones
  \item Despliegue y actualización de aplicaciones en nube
  \item Configuración automática de las aplicaciones
  \item Gestión del almacenamiento
  \item Contenedores. Un mundo enorme
  \item Seguridad
  \item ...
  \end{itemize}
Hay más oportunidades que nunca
\end{frame}

\begin{frame}
  \frametitle{Trabajar en la nube}
  \begin{itemize}
  \item ¿Por qué no trabajar para las empresas de la nube?
  \item Cada vez será más frecuente trabajar en remoto
  \item Oportunidades para trabajar para organizaciones o empresas internacionales
  \item ¿Estamos formando para esos roles?
  \end{itemize}
\end{frame}

\begin{frame}
  \frametitle{}
  \begin{center}
    ``Mientras haya sistemas habrá Administradores de Sistemas''
  \end{center}
\end{frame}
\end{document}
