\documentclass[aspectratio=169]{beamer}
\usepackage[utf8]{inputenc}
\usepackage[T1]{fontenc}

\usetheme{focus}
\definecolor{main}{RGB}{11, 93, 25}

\usecolortheme[RGB={11,93,20}]{structure}
\usepackage[spanish]{babel}

\usepackage{colortbl}
\usepackage{color}
\usepackage{pifont}
\usepackage{ulem}



\usepackage{colortbl}
\usepackage{color}
\usepackage{pifont}
%\usepackage[normalem]{ulem}                                                                 
                                                                                             
\usepackage{listings}

\parskip=12pt

\lstset{basicstyle=\normalsize,
aboveskip=5pt,
basicstyle=\small\ttfamily,
belowskip=5pt
}

\author{Alberto Molina Coballes}
\title{Linux: Atributos de ficheros extendidos}
\institute{IES Gonzalo Nazareno}
\titlegraphic{\includegraphics[width=1.5cm]{cc_by_sa.png}}
\logo{\includegraphics[width=.75cm]{logo_iesgn.png}}
\date{\today}

\definecolor{verde}{rgb}{0,0.73,0}

\begin{document}

\def\braces#1{[#1]}

\begin{frame}[t,plain]
\titlepage
\end{frame}

\begin{frame}[fragile]
  \frametitle{Atributos extendidos de ficheros}
  \begin{itemize}
  \item Pares clave-valor asociados a un fichero, ordenadas en cuatro clases:
    \begin{itemize}
    \item security
    \item system
    \item trusted
    \item user
    \end{itemize}
  \item Las capacidades de núcleo se guardan como atributos extendidos de seguridad
  \item Es necesario que el sistema de ficheros incluya soporte para xattrs
  \item Los programas para utilizar xattrs se incluyen en el paquete \texttt{attr}
  \item La clase ``user'' permite almacenar cualquier información asociada al fichero:
    \begin{lstlisting}
setfattr -n user.checksum -v "3baf9ebce4c664ca8d9e5f6aaafb47fb" \
fichero
    \end{lstlisting}
  \end{itemize}
\end{frame}

\end{document}
