\documentclass[aspectratio=169]{beamer}
\usepackage[utf8]{inputenc}
\usepackage[T1]{fontenc}

\usetheme{focus}
\definecolor{main}{RGB}{11, 93, 25}

\usecolortheme[RGB={11,93,20}]{structure}
\usepackage[spanish]{babel}

\usepackage{colortbl}
\usepackage{color}
\usepackage{pifont}
\usepackage{ulem}



\usepackage{colortbl}
\usepackage{color}
\usepackage{pifont}
%\usepackage[normalem]{ulem}
\usepackage{listings}

\parskip=12pt

\lstset{basicstyle=\normalsize,
aboveskip=5pt,
basicstyle=\small\ttfamily,
belowskip=5pt
}

\author{Alberto Molina Coballes}
\title{ZFS}
\institute{IES Gonzalo Nazareno}
\titlegraphic{\includegraphics[width=1.5cm]{cc_by_sa.png}}
\logo{\includegraphics[width=.75cm]{logo_iesgn.png}}
\date{\today}

\definecolor{verde}{rgb}{0,0.73,0}

\begin{document}

\def\braces#1{[#1]}

\begin{frame}[t,plain]
\titlepage
\end{frame}

\begin{frame}
  \frametitle{El proyecto ZFS}
  \begin{itemize}
  \item 2001: Comienza el desarrollo en Sun Microsystems para Solaris
  \item 2005: Se libera con licencia CDDL y se incluye en
    OpenSolaris. CDDL es incompatible con GPL: ZFS no puede incluirse
    en el kérnel linux
  \item 2006: Se desarrolla ZFS sobre FUSE para sistemas linux
  \item 2010: Oracle compra Sun Microsystems y abandona
    OpenSolaris. ZFS vuelve a ser software cerrado
  \item 2010: Se crea Illumos, fork libre de OpenSolaris
  \item 2013: Se anuncia
    \href{http://open-zfs.org/wiki/Main_Page}{OpenZFS} como fork libre
    de ZFS
  \item 2015: Aparece \href{https://zfsonlinux.org/}{ZFS on Linux},
    versión de OpenZFS para linux
  \end{itemize}
\end{frame}

\begin{frame}
  \frametitle{Características de \sout{ZFS} OpenZFS}
  \begin{itemize}
  \item Es un sistema completo de almacenamiento que no requiere
    otras herramientas
  \item Gestiona los dispositivos de bloques directamente
  \item Incluye su propia implementación de RAID
  \item CoW, deduplicación, instantáneas, compresión, cifrado, soporte
    nativo de nfs, cifs o iscsi, \ldots
  \item Siempre consistente sin necesidad de chequeos
  \item Se autorepara de forma continua
  \item Muy escalable
  \item Exigente en recursos
  \end{itemize}
\end{frame}

\begin{frame}
  \frametitle{Instalación en Debian}
  \begin{itemize}
  \item \url{https://tracker.debian.org/pkg/zfs-linux}
  \item Modificación de los repositorios para incluir la rama contrib
    y el repositorio de backports
  \item Preparación del sistema para compilar módulos
  \item Instalación de los módulos del kérnel: \texttt{spl-dkms} y \texttt{zfs-dkms}
  \item Instalación de las herramientas del espacio de usuario: \texttt{zfsutils-linux}
  \end{itemize}
\end{frame}

\begin{frame}
  \frametitle{Demo}
\end{frame}
\end{document}
