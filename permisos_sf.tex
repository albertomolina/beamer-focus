\documentclass[aspectratio=169]{beamer}
\usepackage[utf8]{inputenc}
\usepackage[T1]{fontenc}

\usetheme{focus}
\definecolor{main}{RGB}{11, 93, 25}

\usecolortheme[RGB={11,93,20}]{structure}
\usepackage[spanish]{babel}

\usepackage{colortbl}
\usepackage{color}
\usepackage{pifont}
\usepackage{ulem}



\usepackage{colortbl}
\usepackage{color}
\usepackage{pifont}
%\usepackage[normalem]{ulem}                                                                 \
                                                                                              
\usepackage{listings}

\parskip=12pt

\lstset{basicstyle=\normalsize,
aboveskip=5pt,
basicstyle=\small\ttfamily,
belowskip=5pt
}

\author{Alberto Molina Coballes}
\title{Linux: Sistemas de ficheros y permisos}
\institute{IES Gonzalo Nazareno}
\titlegraphic{\includegraphics[width=1.5cm]{cc_by_sa.png}}
\logo{\includegraphics[width=.75cm]{logo_iesgn.png}}
\date{\today}

\definecolor{verde}{rgb}{0,0.73,0}

\begin{document}

\def\braces#1{[#1]}

\begin{frame}[t,plain]
\titlepage
\end{frame}

\begin{frame}
  Recordamos los permisos tradicionales UNIX sobre ficheros:
  \begin{itemize}
  \item Esquema ugoa
  \item Permisos especiales (SUID, SGID, sticky bit)
  \item Es una implementación de un sistema de DAC (\textit{Discretionary
      Access Control}) ya que los usuarios pueden modificar los
    permisos (chmod)
  \end{itemize}
\end{frame}

\begin{frame}
  \frametitle{En el inicio fue ugoa}
  \begin{itemize}
  \item Tradicionalmente en UNIX se definen los permisos de ficheros
    para usuario (u), grupo (g), otros (o) o todos (a) 
  \item Los tres permisos básicos son lectura (r), escritura (w) y
    ejecución (x) 
  \item Para borrar un fichero necesitamos permiso de escritura y
    ejecución en el directorio padre 
  \item Un usuario tiene un grupo principal y puede pertenecer a otros
    grupos 
  \item Los permisos iniciales de un fichero se definen mediante la
    orden \texttt{umask} 
  \item Se puede cambiar el grupo principal en una sesión con
    \texttt{newgrp} 
  \item Notación octal: rwx = 111, rw- = 110, etc.
  \end{itemize}
\end{frame}

\begin{frame}[fragile]
  \frametitle{Permisos especiales de ficheros}
  \begin{itemize}
  \item Al establecer \textit{set user identification}
    (\texttt{suid}) sobre un fichero ejecutable, este lo puede
    ejecutar otro usuario con los permisos del propietario.
  \item Si el propietario es superusuario puede ser arriesgado
  \item Ficheros con bit de suid activado:
\begin{verbatim}
find / -perm /4000
\end{verbatim}
  \item Al establecer \textit{set group identification}
    (\texttt{sgid}) sobre un fichero ejecutable, ocurre lo mismo que
    con suid, pero ahora se aplican los permisos del grupo
    propietario 
  \item Ficheros con bit de sgid activado:
\begin{verbatim}
find / -perm /2000
\end{verbatim}
  \end{itemize}
\end{frame}

\begin{frame}
  \frametitle{Permisos especiales de ficheros}
  \begin{itemize}
  \item sgid sobre un directorio: Todos los ficheros que se creen
    heredan el grupo propietario del directorio
  \item suid y sgid se indican con (s)
  \item Al activar el \textit{sticky bit} en un directorio, sólo el
    propietario del fichero podrá borrarlo 
  \item Utilizado en directorios donde varios usuarios pueden escribir
    (por ejemplo /tmp) 
  \item Muy interesante combinado con sgid
  \item sticky bit se indica con (t)
  \end{itemize}
\end{frame}

\end{document}
