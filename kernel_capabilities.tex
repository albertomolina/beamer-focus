\documentclass[aspectratio=169]{beamer}
\usepackage[utf8]{inputenc}
\usepackage[T1]{fontenc}

\usetheme{focus}
\definecolor{main}{RGB}{11, 93, 25}

\usecolortheme[RGB={11,93,20}]{structure}
\usepackage[spanish]{babel}

\usepackage{colortbl}
\usepackage{color}
\usepackage{pifont}
\usepackage{ulem}



\usepackage{colortbl}
\usepackage{color}
\usepackage{pifont}
%\usepackage[normalem]{ulem}                                                                 \
                                                                                              
\usepackage{listings}

\parskip=12pt

\lstset{basicstyle=\normalsize,
aboveskip=5pt,
basicstyle=\small\ttfamily,
belowskip=5pt
}

\author{Alberto Molina Coballes}
\title{Linux kernel POSIX Capabilities}
\institute{IES Gonzalo Nazareno}
\titlegraphic{\includegraphics[width=1.5cm]{cc_by_sa.png}}
\logo{\includegraphics[width=.75cm]{logo_iesgn.png}}
\date{\today}

\definecolor{verde}{rgb}{0,0.73,0}

\begin{document}

\def\braces#1{[#1]}

\begin{frame}[t,plain]
\titlepage
\end{frame}

\begin{frame}
  \frametitle{Introducción}
  \begin{itemize}
  \item Tradicionalmente dos privilegios:
    \begin{itemize}
    \item Procesos privilegiados: Se saltan las comprobaciones de permisos
    \item Procesos no privilegiados: Comprobación estricta de permisos
    \end{itemize}
  \item \textit{Kernel Capabilities}: Mecanismo de seguridad basado en
    el principio de mínimo privilegio, agrupando ciertos privilegios
    en una ``capacidad''
  \item Se le puede asignar a un proceso una capacidad específica a
    nivel del kérnel (\textit{kernel capability})
  \item No son originales de linux: POSIX capabilities, detalladas en
    el borrador (retirado) 1003.1e:
    \footnotesize{\url{http://wt.tuxomania.net/publications/posix.1e/download.html}}
  \end{itemize}
\end{frame}

\begin{frame}
  \frametitle{Lista de capacidades}
  \texttt{man 7 capabilities}\par
  Algunos ejemplos:
  \begin{itemize}
  \item \texttt{CAP\_CHOWN}
  \item \texttt{CAP\_KILL}
  \item \texttt{CAP\_NET\_ADMIN}
  \item \texttt{CAP\_NET\_BIND\_SERVICE}
  \item \texttt{CAP\_NET\_RAW}
  \item \texttt{CAP\_SYS\_ADMIN}
  \item \texttt{CAP\_SYS\_MODULE}
  \item \texttt{CAP\_SYS\_RAWIO}
  \item \texttt{CAP\_SYS\_TIME}
  \end{itemize}
\end{frame}

\begin{frame}
  \frametitle{Conjuntos de capacidades de un ejecutable}
  \begin{itemize}
  \item Se pueden definir los siguientes conjuntos de capacidades a un
    ejecutable
    \begin{description}
    \item[Permitidas(p)] Automáticamente permitidas, independientemente
      de las capacidades heredadas del proceso padre 
    \item[Heredables(i)] Se añaden al proceso junto con las del proceso
      padre para determinar las capacidades permitidas
    \item[Efectivas(e)] Usadas para permitir capacidades de linux en
      aplicaciones que no las soportan directamente
    \end{description}
  \end{itemize}
  \footnotesize{\texttt{\href{http://www.andy-pearce.com/blog/posts/2013/Mar/file-capabilities-in-linux/}{Andy Pearce: File Capabilities In Linux}}}
\end{frame}

\begin{frame}
  \frametitle{Definiendo capacidades}
  \begin{itemize}
  \item Se instala el paquete \texttt{libcap2-bin}
  \item \texttt{setcap}: Define las capacidades de un fichero
  \item \texttt{getcap}: Obtiene las capacidades de un fichero
  \item \texttt{getpcaps}: Lista las capacidades de un proceso
  \end{itemize}
\end{frame}

\end{document}
