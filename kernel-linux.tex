\documentclass{beamer}
\usepackage[utf8]{inputenc}
\usepackage[T1]{fontenc}

\usetheme{focus}
\definecolor{main}{RGB}{11, 93, 25}

\usecolortheme[RGB={11,93,20}]{structure}
\usepackage[spanish]{babel}

\usepackage{colortbl}
\usepackage{color}
\usepackage{pifont}
\usepackage{ulem}



\author{Alberto Molina Coballes}
\title{El kérnel linux}
\institute{IES Gonzalo Nazareno}
\titlegraphic{\includegraphics[width=1.5cm]{cc_by_sa.png}}
\logo{\includegraphics[width=.75cm]{logo_iesgn.png}}
\date{\today}

\definecolor{verde}{rgb}{0,0.73,0}

\begin{document}
\begin{frame}[t,plain]
\titlepage
\end{frame}

\begin{frame} \frametitle{El kérnel linux}
  \begin{itemize}
  \item Características principales
  \item Características de la compilación
  \item Carga del sistema
  \item Manejo de módulos
  \item Compilación de módulos
  \item Compilación del kérnel
  \end{itemize}  
\end{frame}

\begin{frame} \frametitle{Características principales}
  \begin{itemize}
  \item Kérnel del sistema GNU/Linux, licenciado bajo la GNU GPL
  \item Desarrollo colaborativo de miles de personas
  \item Monolítico
  \item LKM: Loadable kernel module
  \item Última versión estable: 5.9.1 (17/10/2020)
  \item \url{kernel.org}
  \item Portado a gran cantidad de arquitecturas, desde pequeños
    dispositivos a grandes supercomputadoras.
  \end{itemize}
\end{frame}

\begin{frame} \frametitle{Características de la compilación}
  \begin{itemize}
  \item El código fuente de rama \textit{vanilla} del núcleo ocupa actualmente 1.1GiB 
  \item Los componentes del kérnel se compilan de dos formas:
    \begin{itemize}
    \item Se incluyen dentro de un fichero ejecutable enlazado
      estáticamente y que habitualmente se denomina \texttt{vmlinuz} o
      \texttt{zImage}
    \item Se compilan individualmente en ficheros objetos con
      extensión \texttt{.ko} que se cargan en memoria a demanda
      (están ubicados en \texttt{/lib/modules})
    \end{itemize}
  \item Soluciones para hardware no detectado en el arranque:
    \begin{itemize}
    \item Se aumenta el tamaño del fichero ejecutable (\texttt{bzImage})
    \item Se montan temporalmente algunos módulos en memoria (\textit{initramfs})
    \end{itemize}
  \item Distribuciones de uso general en sistemas x86 (x86\_64):
    \begin{itemize}
    \item Enorme variedad de hardware
    \item Se incluyen gran cantidad de módulos
    \end{itemize}
  \item Es posible compilar un kérnel para un hardware determinado y
    reducir mucho su tamaño.
  \end{itemize}
\end{frame}

% \begin{frame} \frametitle{Carga del sistema - I}
%   \begin{itemize}
%   \item Se inicia el sistema cargando la BIOS (EFI \ldots WTF?)
%   \item Se realiza la secuencia POST (\textit{Power-On Self-Test})
%   \item Se lee el MBR del disco duro donde está la información de las
%     particiones del disco y el gestor de arranque (normalmente GRUB).
%   \item Se carga en memoria el fichero ejecutable comprimido \texttt{vmlinuz-\ldots}
%   \item Se monta el initramfs (fichero \texttt{initrd-\ldots})
%     $\leftarrow$ opcional
%   \item Se comprueba la memoria, tipo de placa y CPU(s)
%   \item Se activa el sistema Plug and Play
%   \item Se inicializan los dispositivos virtuales (LVM y RAID)
%   \item Se libera la memoria ocupada por el initramfs
%   \item Se ejecuta el proceso \texttt{init} con PID=1
%   \item Se ejecutan los scripts de \texttt{/etc/rcS.d} $\leftarrow$
%     depende de la distro
%   \end{itemize}
%   \tiny{Principales pasos en el arranque de un sistema GNU/Linux instalado en
%     un disco duro de un equipo x86}
% \end{frame}
% \begin{frame} \frametitle{Carga del sistema - II}
%   \begin{itemize}
%   \item Se establece el nombre del equipo (\texttt{hostname})
%   \item Se monta VFS
%   \item Se inicia udevd, que puebla \texttt{/dev} y carga los módulos necesarios
%   \item Se chequea el sistema de ficheros raíz
%   \item Se procesa el fichero \texttt{/etc/modules}
%   \item Se chequean todos los sistemas de ficheros
%   \item Se cargan los parámetros del kérnel especificados en \texttt{/etc/sysctl.conf}
%   \item Se montan todos los sistemas de ficheros
%   \item Se limpian los ficheros temporales \texttt{/tmp},
%     \texttt{/var/run} y \texttt{/var/lock}
%   \item Se levantan las interfaces de red
%   \item Se ejecutan en orden los scripts del resto de niveles de ejecución
%   \end{itemize}
% \end{frame}

\begin{frame} \frametitle{Manejo de módulos}
  La mayoría de los módulos se cargan automáticamente cuando es
  necesario, pero es posible cargarlos o descargarlos manualmente:\par
\vspace{1cm}
  \begin{itemize}
  \item \texttt{lsmod}: Módulos cargados
  \item \texttt{modprobe 'módulo' }: Carga el módulo en memoria
  \item \texttt{modprobe -r 'módulo' }: Descarga el módulo de la memoria
  \item \texttt{find /lib/modules/`uname -r` -type f -iname '*.ko' }: Módulos disponibles
  \item \texttt{modinfo 'módulo' }: Información del módulo
  \item \texttt{depmod}: Actualiza las dependencias de los módulos 
  \end{itemize}
\end{frame}

\begin{frame}
  \frametitle{Parámetros del kérnel}
  \begin{itemize}
  \item Algunos módulos incluyen parámetros que podemos obtener con
    \texttt{modinfo -p <modulo>}
  \item Hay parámetros del kernel que usan valores por defecto salvo
    que los especifiquemos
  \item Parámetros de módulos:
    \begin{itemize}
    \item Directamente en el gestor de arranque
    \item En el fichero /etc/modprobe.d/<modulo>.conf
    \end{itemize}
  \item Parámetros del kérnel
    \begin{itemize}
    \item Directamente sobre /proc/sys
    \item Mediante \texttt{sysctl}
    \end{itemize}
  \end{itemize}
\end{frame}

\begin{frame}[containsverbatim]\frametitle{Compilación de módulos (Debian)}
En algunas ocasiones es necesario compilar un módulo, normalmente
algún controlador de dispositivo que no se encuentra soportado en la
rama oficial del kérnel.
\begin{itemize}
\item Instalamos los paquetes necesarios para compilar:
\begin{verbatim}
# aptitude install build-essential
\end{verbatim}
\item Instalamos los ficheros de cabeceras del kérnel actual:
\begin{verbatim}
# aptitude install linux-headers-`uname -r`
\end{verbatim}
\item Si existe un paquete Debian con el módulo sin
  compilar (ndiswrapper, fuse, madwifi, \ldots) $\Rightarrow$
  Utilizar \texttt{module-assistant}
\item En el resto de casos, se descomprime el paquete que incluye los
  ficheros fuentes del módulo en \texttt{/usr/src} y se siguen las
  instrucciones del fichero \texttt{README} que debe incluir.
\end{itemize}
\end{frame}

\begin{frame}
  \frametitle{DKMS}
  \begin{itemize}
  \item \textit{Dynamic Kernel Module Support}
  \item Compila módulos que están fuera de la rama principal del kérnel
  \item Normalmente se utiliza con módulos que tienen liciencias no
    compatibles con GPL o directamente que no son libres
  \item Debian proporciona varios paquetes listos para usar con DKMS
  \item Recompila los módulos cada vez que se instala un nuevo núcleo
  \end{itemize}
\end{frame}

\begin{frame} \frametitle{Compilación del kérnel (Debian)}
Es poco habitual tener que compilar un núcleo completo, puede ser
necesario cuando se utiliza hardware muy peculiar o como en este caso
simplemente con fines educativos.\par
\end{frame}

\end{document}
