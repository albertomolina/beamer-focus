\documentclass{beamer}
\usepackage[utf8]{inputenc}
\usepackage[T1]{fontenc}

\usetheme{focus}
\definecolor{main}{RGB}{11, 93, 25}

\usecolortheme[RGB={11,93,20}]{structure}
\usepackage[spanish]{babel}

\usepackage{colortbl}
\usepackage{color}
\usepackage{pifont}
\usepackage{ulem}



\usepackage{colortbl}
\usepackage{color}
\usepackage{pifont}
%\usepackage[normalem]{ulem}

\parskip=12pt

\author{Alberto Molina Coballes}
\title{\emph{Red Hat y derivadas}}
\institute{IES Gonzalo Nazareno}
\titlegraphic{\includegraphics[width=1.5cm]{cc_by_sa.png}}
\logo{\includegraphics[width=.75cm]{logo_iesgn.png}}
\date{\today}

\definecolor{verde}{rgb}{0,0.73,0}

\begin{document}

\def\braces#1{[#1]}

\begin{frame}[t,plain]
\titlepage
\end{frame}

\begin{frame}
  \frametitle{Índice}
  \begin{itemize}
  \item Red Hat
  \item Distribuciones derivadas
  \item Principales diferencias con Debian
  \end{itemize}
\end{frame}

\begin{frame}
  \frametitle{Red Hat. Principales hitos.}
  \begin{itemize}
  \item Red Hat Inc. Creada en 1993/1994 (Bob Young y Marc Erwing)
  \item 1994: Se publica Red Hat Linux (RHL)
  \item 1999: \href{https://es.finance.yahoo.com/echarts?s=RHT}{Sale a bolsa}
  \item 2003: Se abandona RHL y surge RHEL
  \item 2003: Aparece Fedora (comunidad apoyada por RH)
  \item 2004: Aparece CentOS (proyecto independiente)
  \item 2014: Red Hat acuerda apoyar a CentOS
  \item 2019: IBM anuncia la compra Red Hat en octubre de 2018 por 34.000 millones de \$
  \end{itemize}
\end{frame}

\begin{frame}
  \frametitle{Red Hat. Modelo de negocio}
  \begin{itemize}
  \item Inicialmente software Unix y Linux
  \item Se centra en soluciones de software libre
  \item Siempre utilizando el término ``open source''
  \item Ofrece soporte, consultoría, formación y certificación en sus tecnologías
  \item Una de las empresas que más contribuye en proyectos de software libre
  \item Trabaja típicamente con grandes clientes
  \end{itemize}
\end{frame}

\begin{frame}
  \frametitle{Red Hat. Principales adquisiciones}
  \begin{description}
  \item[2000] Cygnus
  \item[2003] Sistina (GFS, LVM, DM)
  \item[2006] JBoss
  \item[2008] Qumranet (KVM, SPICE)
  \item[2011] Gluster
  \item[2012] ManageIQ
  \item[2014] Inktank (Ceph)
  \item[2014] e-Novance (OpenStack)
  \item[2015] Ansible
  \item[2018] CoreOS
  \end{description}
\end{frame}

\begin{frame}
  \frametitle{Red Hat. Principales productos}
  \begin{itemize}
  \item Sistemas: RHEL, Satellite
  \item Virtualización, Hiperconvergencia
  \item Cloud: OpenStack, OpenShift
  \item Middleware: JBoss
  \item Almacenamiento: Ceph y Gluster
  \item Automatización: Ansible
  \end{itemize}
  \href{https://www.redhat.com/es/technologies/all-products}{www.redhat.com/es/technologies/all-products}
\end{frame}

\begin{frame}
  \frametitle{Versiones de RHEL}
  \begin{center}
    \begin{tabular}{lll}
      \hline
      \hline
      Años&Versión&Núcleo\\
      \hline
      2002-2005& 2.1 (1-7)& 2.4.9\\
      2003-2007&3 (1-9)&2.4.21\\
      2005-2011&4 (1-9)&2.6.9\\
      2007-2014&5 (1-11)&2.6.18\\
      2010-&6 (1-10-\ldots)&2.6.32\\
      2014-&7 (1-7-\ldots)&3.10\\
      2019-&8 &4.18\\
      \hline
      \hline
    \end{tabular}
  \end{center}
  \url{https://access.redhat.com/articles/3078}
\end{frame}

\begin{frame}
  \frametitle{Fedora}
  \begin{itemize}
  \item Principal producto del ``Fedora Project''
  \item Comunidad soportada por Red Hat
  \item Se publica cada 6 meses (aproximadamente) y se soporta durante 13 meses más.
  \item No existe versión LTS
  \item Incluye software más actualizado que RHEL o CentOS
  \item Última versión estable (Oct 2020): Fedora 33
  \item Fedora Silverblue
  \end{itemize}
\end{frame}

\begin{frame}
  \frametitle{Principales diferencias técnicas con Debian}
  \begin{itemize}
  \item Paquetes rpm en lugar de deb
  \item Gestión de paquetes con rpm en lugar de dpkg
  \item Gestión de repositorios con \sout{yum} dnf en lugar de apt
  \item Repositorios oficiales con muchos menos paquetes
  \item Soporte de hasta 10 años
  \item Sólo son accesibles los paquetes de las versiones estables
  \item Utiliza anaconda en lugar de debian-installer
  \item SELinux activado por defecto
  \item Utlización de firewalld 
  \item Directorio /etc/sysconfig
  \end{itemize}
\end{frame}

\end{document}
