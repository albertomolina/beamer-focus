\documentclass[aspectratio=169]{beamer}
\usepackage[utf8]{inputenc}
\usepackage[T1]{fontenc}

\usetheme{focus}
\definecolor{main}{RGB}{11, 93, 25}

\usecolortheme[RGB={11,93,20}]{structure}
\usepackage[spanish]{babel}

\usepackage{colortbl}
\usepackage{color}
\usepackage{pifont}
\usepackage{ulem}



\usepackage{colortbl}
\usepackage{color}
\usepackage{pifont}
% \usepackage[normalem]{ulem}

\usepackage{listings}

\parskip=12pt

\lstset{basicstyle=\normalsize,
aboveskip=5pt,
basicstyle=\small\ttfamily,
belowskip=5pt
}

\author{Alberto Molina Coballes}
\title{Atributos de ficheros}
\institute{IES Gonzalo Nazareno}
\titlegraphic{\includegraphics[width=1.5cm]{cc_by_sa.png}}
\logo{\includegraphics[width=.75cm]{logo_iesgn.png}}
\date{\today}

\definecolor{verde}{rgb}{0,0.73,0}

\begin{document}

\def\braces#1{[#1]}

\begin{frame}[t,plain]
\titlepage
\end{frame}

\begin{frame}
  \frametitle{Atributos de ficheros}
  \begin{itemize}
  \item Asociados inicialmente a ext?, parcialmente soportados por otros sistemas de ficheros (btrfs, xfs, etc.)
  \end{itemize}
  \begin{columns}
    \column{0.35\textwidth}
    \begin{itemize}
    \item a: append only
    \item c: compressed
    \item d: no dump
    \item e: extent format
    \item i: immutable
    \item j: data journalling
    \item s: secure deletion
    \end{itemize}
    \column{0.45\textwidth}
    \begin{itemize}
    \item t: no tail-merging
    \item u: undeletable
    \item A: no atime updates
    \item C: no copy on write
    \item D: synchronous directory updates
    \item S: synchronous updates
    \item T: top of directory hierarchy
    \end{itemize}
  \end{columns}
\end{frame}

\begin{frame}[fragile]
  \frametitle{Usar atributos de ficheros}
  \begin{itemize}
  \item Se establecen o modifican los atributos de ficheros con \texttt{chattr}, por ejemplo:
    \begin{lstlisting}
chattr +u fichero
    \end{lstlisting}
  \item Se comprueban los atributos que están definidos con \texttt{lsattr}
  \item Ambos programas se incluyen en el paquete \texttt{e2fsprogs}
  \item Algunos atributos hay que establecerlos como root o con la capacidad de núcleo correspondiente
  \end{itemize}
\end{frame}

\end{document}