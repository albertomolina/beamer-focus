\documentclass[aspectratio=169]{beamer}
\usepackage[utf8]{inputenc}
\usepackage[T1]{fontenc}

\usetheme{focus}
\definecolor{main}{RGB}{11, 93, 25}

\usecolortheme[RGB={11,93,20}]{structure}
\usepackage[spanish]{babel}

\usepackage{colortbl}
\usepackage{color}
\usepackage{pifont}
\usepackage{ulem}



\usepackage{colortbl}
\usepackage{color}
\usepackage{pifont}
% \usepackage[normalem]{ulem}                                                                 
\usepackage{hyperref}
\hypersetup{
    colorlinks=true,
    linkcolor=blue,
    filecolor=magenta,      
    urlcolor=cyan,
}

\usepackage{listings}

\parskip=12pt

\lstset{basicstyle=\normalsize,
aboveskip=5pt,
basicstyle=\small\ttfamily,
belowskip=5pt
}

\author{Alberto Molina Coballes\\
José Domingo Muñoz Rodríguez}
\title{Ansible}
\institute{IES Gonzalo Nazareno}
\titlegraphic{\includegraphics[width=1.5cm]{cc_by_sa.png}}
\logo{\includegraphics[width=.75cm]{logo_iesgn.png}}
\date{\today}

\definecolor{verde}{rgb}{0,0.73,0}

\begin{document}

\def\braces#1{[#1]}

\begin{frame}[t,plain]
\titlepage
\end{frame}

\begin{frame}
  \begin{Large}
  \begin{center}
    Introducción    
  \end{center}    
  \end{Large}
\end{frame}

\begin{frame}
  \frametitle{Despliegue tradicional en un servidor}
  \begin{itemize}
  \item Aprovisionamiento del servidor:
    \begin{itemize}
    \item Comprar el servidor o crear la máquina virtual
    \item Instalar y configurar el SO
    \item Instalar y configurar los servicios
    \item Configurar la seguridad
    \end{itemize}
  \item Despliegue de aplicaciones
  \item Documentar todo es la clave
  \item Misma configuración utilizada por años
  \item Escalado vertical, que implica paradas del servidor
  \end{itemize}
\end{frame}

\begin{frame}
  \frametitle{Despliegue ``moderno'' de un servidor}
  \begin{itemize}
  \item Aprovisionamiento de una máquina virtual o contenedor desde una imagen o plantilla
  \item Uso de herramientas de gestión de la configuración:
    \begin{itemize}
    \item Configuración del SO
    \item Instalación y configuración de servicios
    \item Configuración de la seguridad
    \item Actualizaciones
    \end{itemize}
  \item Despliegue de aplicaciones desde un entorno de pruebas idéntico al de producción
  \item Idealmente se utiliza escalado horizontal
  \item Los servidores no tienen por qué mantener la misma configuración mucho tiempo
  \end{itemize}
\end{frame}

\begin{frame}
  \frametitle{Cambio de paradigma: Infraestructura como código}
  Usa tu infraestructura como el software que es:
  \begin{itemize}
  \item Utiliza software de control de versiones
  \item Utiliza un buen editor de textos
  \item Todo legible y con comentarios
  \item Utiliza software de orquestación y de gestión de la configuración
  \item Devops
  \end{itemize}
\end{frame}

\begin{frame}
  \frametitle{Software de orquestación}
  \begin{itemize}
  \item Utilizado para crear escenarios completos con múltiples servidores o contenedores (aprovisionamiento de recursos)
  \item Muy útil en demanda variable de recursos
  \item Muy útil en entornos en los que se cambia continuamente la configuración
  \item Puede incluir funcionalidad de autoescalado
  \end{itemize}
\end{frame}

\begin{frame}
  \frametitle{Software de gestión de la configuración (CMS)}
  Proporciona la gestión y configuración de:
  \begin{itemize}
  \item Aprovisionamiento
  \item Instalación
  \item Configuración
  \item Actualizaciones
  \end{itemize}
\end{frame}

\begin{frame}
  \begin{Large}
    \begin{center}
      CMS de software libre
    \end{center}
  \end{Large}
\end{frame}

\begin{frame}
  \frametitle{Idempotencia}
  \begin{quote}
    Propiedad de ciertas operaciones matemáticas que pueden aplicarse múltiples veces, cambiando el resultado sólo la primera vez
  \end{quote}
  \begin{itemize}
  \item Se utiliza en este ámbito para indicar la diferencia entre los CMS y otras opciones de configuración automátia
  \item Una receta de un CMS se puede ejecutar múltiples veces y el resultado tiene que ser el siempre el mismo
  \end{itemize}
\end{frame}

\begin{frame}
  \frametitle{Puppet}
  \begin{itemize}
  \item Desarrollado por Puppet Labs: \href{https://puppet.com/}{puppet.com}
  \item Escrito en \texttt{C++} y Clojure
  \item Primera versión: 2005
  \item Arquitectura cliente-servidor: Agente en los clientes y servidor con las configuraciones
  \item Los clientes comprueban cada cierto tiempo si es necesario aplicar cambios y en caso necesario los aplican
  \item Configuración mediante manifiestos (manifests) con sintaxis propia
  \item Muchos manifiestos disponibles (Forja de puppet): \href{https://forge.puppet.com/}{forge.puppet.com}
  \end{itemize}
\end{frame}

\begin{frame}
  \frametitle{Chef}
\begin{itemize}
\item Desarrollado por Chef (antes OpsCode): \href{https://www.chef.io/}{chef.io}
\item Escrito en ruby
\item Primera versión: 2009
\item Arquitectura pull: Chef server, chef client y workstation
\item Los clientes comprueban cada cierto tiempo si es necesario aplicar cambios y en caso necesario los aplican
\item Configuración mediante recetas (recipes) y libros de recetas (cookbooks) en ruby
\item Muchos libros de recetas disponibles (chef supermarket): \href{https://supermarket.chef.io/}{supermarket.chef.io}
\end{itemize}
\end{frame}

\begin{frame}
  \frametitle{SaltStack (Salt)}
  \begin{itemize}
  \item Desarrollado por SaltStack: \href{https://saltstack.com/}{saltstack.com}
  \item Escrito en Python
  \item Primera versión: 2011
  \item Master y minions
  \item Descripción de recursos en YAML
  \item Incluye monitorización para respuesta a eventos
  \end{itemize}
\end{frame}

\begin{frame}
  \frametitle{Ansible}
  \begin{itemize}
  \item Desarrollado principalmente por Red Hat (antes ansible): \href{https://www.ansible.com/}{www.ansible.com}
  \item Escrito en Python
  \item Primera versión: 2012
  \item Arquitectura push
  \item No utiliza ningún agente: ssh
  \item Jugadas (plays) y libros de jugadas (playbooks) en YAML
  \end{itemize}
\end{frame}

\begin{frame}
  \frametitle{¿Por qué ansible?}
  \begin{itemize}
  \item Cualquier CMS es una buena opción
  \item Salt y Ansible son más sencillos de aprender y la sintaxis es conocida (YAML)
  \item Ansible no utiliza agentes, sólo ssh (!)
  \item Fácil de instalar (disponible en pypi)
  \item Comunidad muy activa
  \item Más cercano a la forma de trabajar de administradores de sistemas
  \end{itemize}
\end{frame}

\begin{frame}
  \begin{Large}
    \begin{center}
      Ansible
    \end{center}
  \end{Large}
\end{frame}

\begin{frame}
  \frametitle{Instalación}
  \begin{itemize}
  \item \href{https://docs.ansible.com/ansible/latest/installation\_guide/intro_installation.html}{Installing Ansible}
  \item Está empaquetado para varias distros:
    \begin{itemize}
    \item \href{https://tracker.debian.org/pkg/ansible}{tracker.debian.org/pkg/ansible}
    \item \href{https://packages.ubuntu.com/ansible}{packages.ubuntu.com/ansible}
    \end{itemize}
  \item Es muy sencillo hacerlo en un entorno virtual de python con pip
  \end{itemize}
\end{frame}

\begin{frame}
  \frametitle{Configuración}
  \begin{itemize}
  \item Puede definirse en diferentes ubicaciones, con la siguiente orden de prioridades:
    \begin{itemize}
    \item \texttt{ANSIBLE\_CONFIG} (variable de entorno)
    \item \texttt{ansible.cfg} (directorio actual)
    \item \texttt{.ansible.cfg} (en el directorio home)      
    \item \texttt{/etc/ansible/ansible.cfg}
    \end{itemize}
  \item \textbf{Recomendación:} Un repositorio git por configuración con su \texttt{ansible.cfg}
  \item ¿Qué podemos definir aquí? \href{https://docs.ansible.com/ansible/latest/installation\_guide/intro_configuration.html}{Configuring Ansible}
  \end{itemize}
\end{frame}

\begin{frame}[fragile]
  \frametitle{Configuración}
  Ejemplo de configuración:
\begin{verbatim}
[defaults]
inventory = ansible_inventory.ini
remote_user = debian
private_key_file = /home/user/.ssh/private_key
host_key_checking = False
\end{verbatim}
\end{frame}

\begin{frame}[fragile]
  \frametitle{Plugins y módulos}
  \begin{itemize}
  \item Se pueden ejecutar directamente o configurándolos en libros de jugadas (\textit{playbooks})
  \item Cada módulo puede recibir parámetros (obligatorios u opcionales)
  \item Ansible evoluciona rápido: Módulos o su formato pasa a estar obsoleto
  \item Incluídos directamente (\textit{built-in}) o adicionales a través de \href{https://galaxy.ansible.com/}{Ansible Galaxy}
  \item \href{https://docs.ansible.com/ansible/latest/collections/all_plugins.html}{Indexes of all modules and plugins}
  \item Un ejemplo:
\begin{verbatim}
ansible controller -m user -a "name=alberto group=adm"
\end{verbatim}
  \end{itemize}
\end{frame}

\begin{frame}
  \frametitle{Libros de jugadas (\textit{playbooks})}
  \begin{itemize}
  \item Los libros de jugadas agrupan jugadas (\textit{plays})
  \item Cada jugada contiene una o varias tareas (y otros elementos)
  \item Las tareas utilizan módulos
  \item Se ejecutan secuencialmente
  \item Están escritos en YAML
  \end{itemize}
\end{frame}

\begin{frame}
  \begin{itemize}
  \item Ansible es fácil de aprender: aprende practicando
  \item Instálalo, configura un fichero de inventario y practica
  \item Empieza con tareas sencillas y módulos
  \item Cuando te familiarices con esos elementos, continúa:
    \begin{itemize}
    \item Manejadores (\textit{handlers})
    \item Variables: Host, Group, facts, \ldots
    \item Ejecución condicionada
    \item Bucles
    \item Roles
    \item Buenas prácticas y reutilización
    \end{itemize}
  \end{itemize}
\end{frame}
\end{document}
