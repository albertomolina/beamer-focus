\documentclass[aspectratio=169]{beamer}
\usepackage[utf8]{inputenc}
\usepackage[T1]{fontenc}

\usetheme{focus}
\definecolor{main}{RGB}{11, 93, 25}

\usecolortheme[RGB={11,93,20}]{structure}
\usepackage[spanish]{babel}

\usepackage{colortbl}
\usepackage{color}
\usepackage{pifont}
\usepackage{ulem}



\usepackage{colortbl}
\usepackage{color}
\usepackage{pifont}
%\usepackage[normalem]{ulem}
\usepackage{listings}

\parskip=12pt

\lstset{basicstyle=\normalsize,
aboveskip=5pt,
basicstyle=\tiny\ttfamily,
belowskip=5pt
}

\author{Alberto Molina Coballes}
\title{Sistemas de Ficheros en Linux. Funcionalidades avanzadas}
\institute{IES Gonzalo Nazareno}
\titlegraphic{\includegraphics[width=1.5cm]{cc_by_sa.png}}
\logo{\includegraphics[width=.75cm]{logo_iesgn.png}}
\date{\today}

\definecolor{verde}{rgb}{0,0.73,0}

\begin{document}

\def\braces#1{[#1]}

\begin{frame}[t,plain]
\titlepage
\end{frame}

\begin{frame}
  \frametitle{Funcionalidades avanzadas}
  \begin{itemize}
  \item \textit{Copy on Write} (CoW)
  \item Deduplicación
  \item Cifrado
  \item Compresión
  \item Gestión de volúmenes
  \item Instantáneas (\textit{snapshots})
  \item Redundancia (RAID)
  \end{itemize}
  \small{\url{https://en.wikipedia.org/wiki/Comparison_of_file_systems}}
\end{frame}

\begin{frame}
  \frametitle{Copy on Write}
  \begin{itemize}
  \item Se descomponen los datos a almacenar en diferentes bloques
  \item Al crear la copia solo se crea un nuevo puntero que apunta al
    conjunto de datos originales
  \item Cuando los datos copiados se modifican, se van creando nuevos
    bloques con las modificaciones
  \item Las versiones divergen a medida que van cambiando
  \end{itemize}
\end{frame}

\begin{frame}
  \frametitle{Deduplicación}
  \begin{itemize}
  \item Utiliza las propiedades CoW de un sistema de ficheros
  \item Identifica bloques idénticos y los reorganiza con CoW
  \item Permite aprovechar la funcionalidad CoW sin intervención del
    usuario
  \item Ejemplo: En un directorio con las cuentas de correo de todos
    los usuarios de la organización se guarda solo una copia del
    fichero adjunto enviado a todas las cuentas
  \end{itemize}
  \small{\url{https://strugglers.net/~andy/blog/2017/01/10/xfs-reflinks-and-deduplication/}}
\end{frame}

\begin{frame}
  \frametitle{Cifrado}
  Permite el cifrado de ficheros al vuelo sin utilizar software adicional
\end{frame}

\begin{frame}
  \frametitle{Compresión}
  Almacena los ficheros comprimidos para optimizar el uso del espacio
\end{frame}

\begin{frame}
  \frametitle{Gestión de volúmenes}
  \begin{itemize}
  \item Equivalente a LVM
  \item Permite gestionar volúmenes y sistemas de ficheros de forma
    independiente de los dispositivos de bloques físicos
  \item Un dispositivo de bloques físico puede contener varios
    volúmenes o un volumen puede estar distribuido en varios
    dispositivos de bloques físicos
  \end{itemize}
\end{frame}

\begin{frame}
  \frametitle{Instantáneas (snapshots)}
  \begin{itemize}
  \item Utiliza instantáneas de forma nativa
  \end{itemize}
\end{frame}

\begin{frame}
  \frametitle{Sumas de comprobación (checksums)}
  Utilizadas para verificar la integridad de los ficheros
\end{frame}

\begin{frame}
  \frametitle{Redundancia (RAID)}
  \begin{itemize}
  \item RAID software nativo en el sistema de ficheros
  \item Sin necesidad de usar mdadm
  \end{itemize}
\end{frame}

\begin{frame}
  \frametitle{Comparativa}
  \begin{center}
    \begin{tabular}[center]{|l|cccccccc|}
      \hline
      &CoW&Dedup&Cifr&Comp&Vols&Snap&cksum&RAID\\
      \hline
      \rowcolor[rgb]{1,0.95,0.77}
      ext4
      & {\color{red}\ding{55}}
          & {\color{red}\ding{55}}
                & {\color{verde}\ding{51}}
                        & {\color{red}\ding{55}}
                              & {\color{red}\ding{55}}
                                   & {\color{red}\ding{55}}
                                          & {\color{red}\ding{55}}
                                                   & {\color{red}\ding{55}}\\
      jfs
      & {\color{red}\ding{55}}
          & {\color{red}\ding{55}}
                & {\color{verde}\ding{51}}
                        & {\color{red}\ding{55}}
                              & {\color{red}\ding{55}}
                                   & {\color{red}\ding{55}}
                                          & {\color{red}\ding{55}}
                                                   & {\color{red}\ding{55}}\\
      \rowcolor[rgb]{1,0.95,0.77}
      xfs
      & {\color{verde}\ding{51}}
          & {\color{verde}\ding{51}}
                & {\color{red}\ding{55}}
                        & {\color{red}\ding{55}}
                              & {\color{red}\ding{55}}
                                   & {\color{red}\ding{55}}
                                          & {\color{red}\ding{55}}
                                                   & {\color{red}\ding{55}}\\
      zfs
      & {\color{verde}\ding{51}}
          & {\color{verde}\ding{51}}
                & {\color{verde}\ding{51}}
                        & {\color{verde}\ding{51}}
                              & {\color{verde}\ding{51}}
                                   & {\color{verde}\ding{51}}
                                          & {\color{verde}\ding{51}}
                                                   & {\color{verde}\ding{51}}\\
      \rowcolor[rgb]{1,0.95,0.77}
      btrfs
      & {\color{verde}\ding{51}}
          & {\color{verde}\ding{51}}
                & {\color{red}\ding{55}}
                        & {\color{verde}\ding{51}}
                              & {\color{verde}\ding{51}}
                                   & {\color{verde}\ding{51}}
                                          & {\color{verde}\ding{51}}
                                                   & {\color{verde}\ding{51}}\\
      \hline
    \end{tabular}
  \end{center}
  \small{\url{https://en.wikipedia.org/wiki/Comparison_of_file_systems}}
\end{frame}
\end{document}
