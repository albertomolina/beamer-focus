\documentclass{beamer}
\usepackage[utf8]{inputenc}
\usepackage[T1]{fontenc}

\usetheme{focus}
\definecolor{main}{RGB}{11, 93, 25}

\usecolortheme[RGB={11,93,20}]{structure}
\usepackage[spanish]{babel}

\usepackage{colortbl}
\usepackage{color}
\usepackage{pifont}
\usepackage{ulem}



\usepackage{svg}

\author{Alberto Molina Coballes}
\title{\centering{\includegraphics[width=2cm]{openlogo-debian.png}}}
\institute{IES Gonzalo Nazareno}
\titlegraphic{\includegraphics[width=1.5cm]{cc_by_sa.png}}
\logo{\includegraphics[width=.75cm]{logo_iesgn.png}}
\date{\today}

\definecolor{verde}{rgb}{0,0.73,0}

\begin{document}
\begin{frame}[t,plain]
\titlepage
\end{frame}

\begin{frame} \frametitle{El proyecto Debian}
  \begin{block}{El proyecto}
    A diferencia de otras distribuciones, Debian es un proyecto de
    software libre sostenido por una comunidad
  \end{block}
  \begin{block}{Contrato Social}
    \begin{itemize}
    \item Debian permanecerá 100\% libre
    \item Contribuiremos a la comunidad de software libre
    \item No ocultaremos los problemas
    \item Nuestra prioridad son nuestros usuarios y el software libre
    \item Trabajos que no siguen nuestros estándares de software libre
      (contrib y non-free)
    \end{itemize}
  \end{block}
\end{frame}

\begin{frame}\frametitle{Características principales}
  \begin{itemize}
  \item Es una de las grandes distribuciones
  \item Muchas otras distros derivan de Debian
  \item Modelo de desarrollo completamente abierto
  \item Formato de paquete \texttt{.deb}
  \item Publicación regular de versiones estables (2 años aprox.)
  \end{itemize}
\end{frame}

\begin{frame}\frametitle{Ramas}
  \begin{block}{\textit{stable}}
    Versión estable y recomendada para entornos en producción. \textbf{buster}
  \end{block}
  \begin{block}{\textit{testing}}
    Versión en desarrollo y futura estable. \textbf{bullseye}
  \end{block}
  \begin{block}{\textit{unstable}}
    Versión inestable, donde se suben las nuevas versiones que luego pasan a testing. \textbf{sid}
  \end{block}
  \begin{block}{\textit{experimental}}
    Versión para nuevos paquetes o versiones que introducen cambios.
  \end{block}
\end{frame}

\begin{frame}{Formato paquete binario \texttt{deb}}
  \begin{block}{Archivo ar}
    \begin{itemize}
    \item debian-binary (Versión, actualmente 2.0)
    \item control.tar.xz (Metadatos y scripts de mantenimiento)
    \item data.tar.xz (Contenido)
    \end{itemize}
  \end{block}
  \begin{itemize}
  \item Los paquetes están firmados por el mantenedor con una clave
    GPG que está en debian-keyring
  \item Los paquetes de una rama deben usar la misma versión de una
    dependencia
  \end{itemize}
\end{frame}

\begin{frame}\frametitle{Ciclo de desarrollo}
  \begin{block}{Desarrollo}
  El desarrollo en debian es abierto y cualquiera puede participar
  \end{block}  
\begin{itemize}
\item Forja: Principalmente \url{https://salsa.debian.org}
\item Info del paquete: \url{https://tracker.debian.org}
\item Errores y sugerencias: \url{https://bugs.debian.org}
\item Listas de correo: \url{https://lists.debian.org}
\item IRC: \url{https://irc.debian.org}
\end{itemize}
\end{frame}

\begin{frame}\frametitle{Versiones publicadas de debian}
\resizebox{.9\textwidth}{!}{%
  \begin{tabular}{cccccc}
    \hline
    Versión&Nombre&Lanzamiento&Arquitecturas&Paquetes&Soporte\\
    \hline
    \hline
    1.1&Buzz&17/6/1996&1&474&1996\\
    1.2&Rex&12/12/1996&1&848&1996\\
    1.3&Bo&2/6/1997&1&974&1997\\
    2.0&Hamm&24/7/1998&2&$\sim 1500$&1998\\
    2.1&Slink&9/3/1999&4&$\sim 2250$&2000-02\\
    2.2&Potato&15/8/2000&6&$\sim 3900$&2003-04\\
    3.0&Woody&19/7/2002&11&$\sim 8500$&2006-08\\
    3.1&Sarge&6/6/2005&11&$\sim 15400$&2008-10\\
    4&Etch&8/4/2007&11&$\sim 18000$&2010-12\\
    5&Lenny&14/2/2009&12&$\sim 23000$&2012-02\\
    6&Squeeze&6/2/2011&9+2&$\sim 29000$&2016\\
    7&Wheezy&4/5/2013&11+2&$\sim 36000$&2018\\
    8&Jessie&25/4/2015&10&$\sim 43000$&30/6/2020\\
    9&Stretch&17/6/2017&10&$\sim 51687$&6/2022\\
    10&Buster&6/7/2019&10&> 59000&2024\\
    \hline
    \hline
  \end{tabular}
}
\footnotesize{\url{https://en.wikipedia.org/wiki/Debian}}
\end{frame}
\begin{frame}
  \frametitle{Manejo de paquetes}
  \begin{itemize}
  \item El manejo de paquetes se realiza a bajo nivel con
    \texttt{dpkg}
  \item Por encima de dpkg funciona el sistema APT (\textit{Advanced
      Packaging Tool}), interfaz única para la descarga y gestión de
    versiones y dependencias
  \item Diferentes programas pueden manejar APT: \texttt{apt-get,
      apt-cache, apt, aptitude, synaptic,} \ldots
  \item Es imprescindible dominar el manejo de \texttt{dpkg} y APT
  \end{itemize}
\end{frame}

\begin{frame}
  \frametitle{Repositorios (\texttt{/etc/apt/sources.list})}
  \begin{block}{Rama \texttt{stable} con componente main}
    deb http://deb.debian.org/debian buster main
  \end{block}
  \begin{block}{Actualizaciones de seguridad de \texttt{stable}}
    deb http://deb.debian.org/debian-security buster/updates main
  \end{block}
  \begin{block}{Paquetes que necesitan modificarse frecuentemente}
    deb http://deb.debian.org/debian buster-updates main  
  \end{block}
  \begin{block}{Paquetes disponibles en \texttt{stable} con versión de
      \texttt{testing}}
    deb http://deb.debian.org/debian buster-backports main  
  \end{block}
  Además de \texttt{main}, existen \texttt{contrib} y \texttt{non-free}
\end{frame}
\end{document}
