\documentclass[aspectratio=169]{beamer}
\usepackage[utf8]{inputenc}
\usepackage[T1]{fontenc}

\usetheme{focus}
\definecolor{main}{RGB}{11, 93, 25}

\usecolortheme[RGB={11,93,20}]{structure}
\usepackage[spanish]{babel}

\usepackage{colortbl}
\usepackage{color}
\usepackage{pifont}
\usepackage{ulem}



\usepackage{colortbl}
\usepackage{color}
\usepackage{pifont}
%\usepackage[normalem]{ulem}                                                                  
\usepackage{listings}

\parskip=12pt

\lstset{basicstyle=\normalsize,
aboveskip=5pt,
basicstyle=\small\ttfamily,
belowskip=5pt
}

\author{Alberto Molina Coballes}
\title{systemd}
\institute{IES Gonzalo Nazareno}
\titlegraphic{\includegraphics[width=1.5cm]{cc_by_sa.png}}
\logo{\includegraphics[width=.75cm]{logo_iesgn.png}}
\date{\today}

\definecolor{verde}{rgb}{0,0.73,0}

\begin{document}

\def\braces#1{[#1]}

\begin{frame}[t,plain]
\titlepage
\end{frame}

\begin{frame}
  \frametitle{Orígenes de systemd}
  \begin{itemize}
  \item Se plantea como sustituto de System V init (PID=1)
  \item Para obtener un arranque rápido en paralelo de los servicios en el espacio de usuario
  \item Comienza su desarrollo en 2010
  \item Compite inicialmente con upstart que comenzó a desarrollarse en 2006
  \item Englobado en el proyecto freedesktop.org: \url{https://www.freedesktop.org/wiki/Software/systemd/}
  \end{itemize}
\end{frame}

\begin{frame}
  \frametitle{Systemd ganó}
  \begin{itemize}
  \item upstart fue inicialmente adoptado por muchas distros
  \item Debian: \href{https://lists.debian.org/debian-devel/2013/10/msg00651.html}{Proposal: switch init system to systemd or upstart} y \href{https://wiki.debian.org/Debate/initsystem/upstart}{Debate initsystem upstart}
  \item Adopción de systemd:
    \begin{itemize}
    \item Fedora: Mayo 2011
    \item Arch Linux: Octubre 2012
    \item RHEL y CentOS: Junio/Julio 2014
    \item Ubuntu: Abril 2015
    \item Debian: Abril 2015
    \end{itemize}
  \end{itemize}
  \href{http://www.markshuttleworth.com/archives/1316}{Mark Shuttleworth: Losing graciously}
\end{frame}

\begin{frame}
  \frametitle{Funcionalidades iniciales}
  \begin{itemize}
  \item Sustituye \texttt{/sbin/init}
  \item Sustituye los niveles de ejecución (\textit{run-levels})
  \item Controla los demonios (olvídate de \texttt{/etc/init.d/...} o service)
  \item Desaparece \texttt{/etc/inittab}
  \item Sustituye \texttt{/etc/fstab}
  \item Se encarga de borrar los directorios temporales
  \item Gestiona los servicios en paralelo
  \item Gestiona los procesos con cgroups
  \item Soporta instantáneas y restauración
  \end{itemize}
  \url{http://0pointer.de/blog/projects/why.html}
\end{frame}

\begin{frame}
  \frametitle{Pero ...}
  Los desarrolladores de systemd pensaron que una vez puestos a sustituir se podría seguir controlando:
  \begin{itemize}
  \item Las sesiones de usuario y las TTYs (systemd-logind)
  \item Syslog (systemd-journald)
  \item Dispositivos (systemd-udevd)
  \item Variables del kérnel (systemd-sysctl)
  \item Sesiones de usuario (systemd-user-sessions)
  \item Puntos de montaje (*.mount)
  \item Control de eventos como alternativa a cron (*.timer)
  \item Hora del sistema (systemd-timedated)
  \item Contenedores (systemd-nspawn)
  \end{itemize}
  Además consideran que systemd es un sistema en continuo desarrollo
\end{frame}

\begin{frame}
  \frametitle{Y algunos todavía no utilizados por todas las distros}
  \begin{itemize}
  \item Interfaces de red (systemd-networkd)
  \item Chequeo del sistema de ficheros (systemd-fsck)
  \item Resolución de nombres (systemd-resolved)
  \item Volcados de memoria (systemd-coredump y coredumpctl)
  \item Cuotas (systemd-quotacheck)
  \item Homes (systemd-homed)
  \item Y lo que te rondaré morena
  \end{itemize}
\end{frame}

\begin{frame}
  \frametitle{Críticas}
  Systemd ha recibido feroces críticas:
  \begin{itemize}
  \item No solo sustituye init, se ha convertido en un supersistema interno que está reemplazando muchos componentes esenciales
  \item Rompe con la filosofía UNIX de pequeñas aplicaciones que realizan muy bien una tarea interactuando entre sí
  \item Incluye cada vez más dependencias que hacen difícil tener un sistema sin systemd
  \item Es exclusivo de Linux y no utilizable en otros Unices
  \item Incluso ha habido forks: \url{https://devuan.org/}
  \end{itemize}
\end{frame}

\begin{frame}
  \frametitle{Características principales}
  \begin{itemize}
  \item Activación de servicios basados en sockets
  \item Activación de servicios basados en D-Bus 
  \item Activación de servicios basados en dispositivos
  \item Activación de servicios basados en ficheros
  \item Montaje y automontaje
  \item Compatible con SysV init
  \end{itemize}
  \href{https://access.redhat.com/documentation/en-us/red_hat_enterprise_linux/7/html/system_administrators_guide/chap-managing_services_with_systemd}{Red Hat System Administrator's Guide}
\end{frame}

\begin{frame}[fragile]
  \frametitle{Conceptos de systemd. Unidades}
  \begin{description}
  \item[Unidad (\textit{unit})] Elemento básico de systemd asociado a un fichero de configuración
  \end{description}
  La instrucción principal para interacturar con systemd es \texttt{systemctl}:
  \begin{itemize}
  \item Información de systemd:
    \begin{lstlisting}
      systemctl status
    \end{lstlisting}
  \item Unidades que han fallado:
    \begin{lstlisting}
      systemctl --failed [--all]
    \end{lstlisting}
  \item Listar unidades en memoria:
    \begin{lstlisting}
      systemctl list-units
    \end{lstlisting}
  \item Listar unidades instaladas:
    \begin{lstlisting}
      systemctl list-unit-files
    \end{lstlisting}
  \end{itemize}
\end{frame}
  
\begin{frame}
  \frametitle{Conceptos de systemd. Unidades}
  Las unidades se ubican en diferentes directorios:
  \begin{itemize}
  \item \texttt{/lib/systemd/system/} Al instalar los diferentes paquetes, estén o no habilitadas
  \item \texttt{/run/systemd/} Creadas durante la ejecución del sistema. Tiene prioridad sobre el anterior
  \item \texttt{/etc/systemd/system/} Unidades habilitadas
  \end{itemize}
\end{frame}

\begin{frame}[fragile]
  \frametitle{Conceptos de systemd. Servicios}
  Las unidades encargadas de la gestión de servicios tienen la extensión \texttt{.service} y se gestionan con \texttt{systemctl}:
  \begin{lstlisting}
    systemctl start foo | foo.service 
    systemctl stop foo | foo.service 
    systemctl restart foo | foo.service 
    systemctl status foo | foo.service 
    systemctl enable foo | foo.service 
    systemctl disable foo | foo.service 
    systemctl mask foo | foo.service 
    systemctl unmask foo | foo.service 
    systemctl is-enabled foo | foo.service
    systemctl is-active foo | foo.service
    systemctl is-failed foo | foo.service
  \end{lstlisting}
\end{frame}

\begin{frame}[fragile]
  \frametitle{Conceptos de systemd. Targets}
  \begin{description}
  \item[\textit{Target}] Conjunto de unidades. Los niveles de ejecución se sustituyen en systemd por \textit{targets}, pero se usan con más fines. Cada distro puede definir sus propios \textit{targets}
    \begin{itemize}
    \item \textit{basic.target}: Unidades para el inicio del sistema
    \item \textit{multi-user.target}: Agrupa la mayoría de los servicios no esenciales. Equivale a los niveles de ejecución 2-4
    \item \textit{graphical.target}: Agrupa los servicios del entorno gráfico. Equivalente al nivel de ejecución 5
    \end{itemize}
  \end{description}
  \begin{lstlisting}
    systemctl get-default
    systemctl list-units --type target [--all]
    systemctl set-default nombre.target
    systemctl isolate nombre.target
    systemctl [--no-wall] rescue
    systemctl [--no-wall] emergency
  \end{lstlisting}
\end{frame}

\begin{frame}[fragile]
  \frametitle{Conceptos de systemd. Porciones}
  \begin{description}
  \item[Porción (\textit{slice})] Utilizado para gestionar grupos de procesos creando nodos en el árbol de Linux cgroups, por lo que se pueden aplicar límites a los recursos utilizados.
    \begin{itemize}
    \item \texttt{system.slice}: Utilizado por defecto por servicios
    \item \texttt{user.slice}: Jerarquía de porciones de recursos de usuarios (los hijos se representan con \texttt{``-''})
    \item \texttt{user-[].slice}: Sesiones de usuarios gestionadas por systemd-logind
    \end{itemize}
    \begin{lstlisting}
      systemctl list-units --type slice [--all]
    \end{lstlisting}
  \end{description}
\end{frame}

\begin{frame}[fragile]
  \frametitle{Conceptos de systemd. Puntos de montaje}
  \begin{description}
  \item [Punto de montaje (\textit{mount})] Define unidades para los puntos de montaje de los sistemas de ficheros en unidades con extensión \texttt{.mount}. Sustituye así las entradas de \texttt{/etc/fstab}, pero por compatibilidad permite definir allí las entradas que se traducen en tiempo de ejecución a unidades en \texttt{/run/systemd}
  \item [\textit{automount}] Para montaje automático (requiere una unidad .mount y otra .automount)
  \item[Intercambio (\textit{swap})] Define unidades para los dispositivos o ficheros de \textit{swap}. Al igual que en el caso anterior, pueden seguir definiéndose en \texttt{/etc/fstab} y se crean las correspondientes unidades durante la ejecución del sistema.
  \end{description}
\end{frame}

\begin{frame}[fragile]
  \frametitle{Conceptos. Temporizadores}
  \begin{description}
  \item[Temporizador (\textit{timer})] Como alternativas a la definición de procesos de \texttt{cron}. Hay de dos tipos:
    \begin{itemize}
    \item Definidos conforme a fechar fijas, como los \textit{cronjobs}
    \item Definido por un intervalo de tiempo tras un evento.
    \end{itemize}
  \end{description}
  \begin{lstlisting}
 systemctl list-timers
NEXT     LEFT          LAST     PASSED       UNIT               ACTIVATES
Sun ...  3h 35min left Sun ...  10h ago      apt-daily.timer    apt-daily.service
Mon ...  18h left      Sun ...  4h 47min ago apt-daily-upgra... apt-daily-upgrade.service
Mon ...  22h left      Sun ...  1h 29min ago systemd-tmpfile... systemd-tmpfiles-clean.service
  \end{lstlisting}
\end{frame}

\begin{frame}[fragile]
  \frametitle{Apagar/suspender/hibernar/reiniciar}
  systemd también controla las funciones de apagado y reinicio, junto con la suspensión (guarda estado en memoria) o la hibernación (guarda el estado en disco) del sistema:
  \begin{lstlisting}
    systemctl [--no-wall] [-f] halt 
    systemctl [--no-wall] [-f] poweroff 
    systemctl [--no-wall] [-f] reboot
    systemctl suspend
    systemctl hibernate
    systemctl hybrid-sleep
  \end{lstlisting}
  Siguen existiendo las instrucciones clásicas \texttt{poweroff}, \texttt{shutdown}, etc. dentro del paquete \texttt{systemd-sysv} pero realizan sus funciones a través de systemd.
\end{frame}

\begin{frame}[fragile]
  \frametitle{timedatectl}
  Instrucción utilizada para consultar y fijar la hora del sistema.
  \begin{lstlisting}
timedatectl 
      Local time: dom 2017-12-10 12:14:57 CET
  Universal time: dom 2017-12-10 11:14:57 UTC
        RTC time: dom 2017-12-10 11:14:57
       Time zone: Europe/Madrid (CET, +0100)
 Network time on: yes
NTP synchronized: yes
 RTC in local TZ: no    
  \end{lstlisting}
  Incluye ntp integrado a través de \texttt{systemd-timesyncd.service}
\end{frame}

\begin{frame}[fragile]
  \frametitle{loginctl}
  \begin{description}
  \item[Puesto (\textit{seat})] Hardware asociado a un puesto de trabajo
  \item[Sesión] Definida por el tiempo entre el ingreso y la salida del usuario
  \end{description}
  \texttt{systemd-logind} permite definir sistemas multisesión y/o multipuesto.\par
  \texttt{loginctl} es la instrucción para controlar las sesiones, puestos y demás aspectos relacionados:
  \begin{lstlisting}
    loginctl list-users
    loginctl list-seats
    loginctl list-sessions
    loginctl user-status [USER]
    ...
  \end{lstlisting}
\end{frame}
\end{document}
