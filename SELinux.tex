\documentclass[aspectratio=169]{beamer}
\usepackage[utf8]{inputenc}
\usepackage[T1]{fontenc}

\usetheme{focus}
\definecolor{main}{RGB}{11, 93, 25}

\usecolortheme[RGB={11,93,20}]{structure}
\usepackage[spanish]{babel}

\usepackage{colortbl}
\usepackage{color}
\usepackage{pifont}
\usepackage{ulem}



\usepackage{colortbl}
\usepackage{color}
\usepackage{pifont}
\usepackage{soul}
%\usepackage[normalem]{ulem}
                                                                                              
\usepackage{listings}

\parskip=12pt

\lstset{basicstyle=\normalsize,
aboveskip=5pt,
basicstyle=\small\ttfamily,
belowskip=5pt
}

\author{Alberto Molina Coballes}
\title{SELinux}
\institute{IES Gonzalo Nazareno}
\titlegraphic{\includegraphics[width=1.5cm]{cc_by_sa.png}}
\logo{\includegraphics[width=.75cm]{logo_iesgn.png}}
\date{\today}

\definecolor{verde}{rgb}{0,0.73,0}

\begin{document}

\def\braces#1{[#1]}

\begin{frame}[t,plain]
\titlepage
\end{frame}

\begin{frame}[fragile]
  \frametitle{SELinux}
  \begin{itemize}
  \item Security-Enhanced Linux (SELinux). Módulos del kérnel que
    implementan principalmente MAC (\textit{Mandatory Access Control})
  \item Además proporciona Role-Based Access Control (RBAC), Type
    Enforcement (TE) y Multi-Level Security (MLS)
  \item El kérnel consulta a SELinux si un proceso está autorizado o no
  \item Desarrollado inicialmente por NSA
  \item Actualmente desarrollado principalmente por Red Hat
  \item Habilitado por defecto en RHEL y CentOS, opcional en otras distribuciones
  \end{itemize}
  \begin{lstlisting}
sudo getenforce
Enforcing
  \end{lstlisting}
\end{frame}

\begin{frame}[fragile]
  \frametitle{SELinux. Modos}
  \begin{description}
  \item[Enforcing] Modo por defecto. SELinux aplicado y denegando cualquier proceso no permitido de forma explícita
    \begin{lstlisting}
setenforce 1
    \end{lstlisting}
  \item[Permissive] SELinux habilitado, registrando cualquier proceso no permitido en logs, pero no denegando
    \begin{lstlisting}
setenforce 0
    \end{lstlisting}
\item[Disabled] SELinux deshabilitado completamente
\end{description}
Es posible pasar un dominio específico a modo permisivo:
\end{frame}

\begin{frame}[fragile]
  \frametitle{Tipos de política}
  \begin{description}
  \item[MLS] \textit{Multi Level Security} utilizado en entornos complejos 
  \item[Targeted] Solo procesos seleccionados se ejecutan en un
    dominio confinado, mientras que el resto lo hace en uno no confinado
    \begin{itemize}
    \item Modo por defecto en RHEL
    \item Los procesos que escuchan peticiones a través de la red
      suelen estar confinados (servicios)
    \end{itemize}
  \end{description}
  \begin{lstlisting}[basicstyle=\scriptsize\ttfamily]
[root@selinux centos]# sestatus 
SELinux status:                 enabled
SELinuxfs mount:                /sys/fs/selinux
SELinux root directory:         /etc/selinux
Loaded policy name:             targeted <-----
Current mode:                   enforcing
Mode from config file:          enforcing
Policy MLS status:              enabled
Policy deny_unknown status:     allowed
Memory protection checking:     actual (secure)
Max kernel policy version:      31
  \end{lstlisting}
\end{frame}

\begin{frame}
  \frametitle{Contextos}
  \begin{itemize}
  \item Se utilizan reglas (políticas) para autorizar o prohibir cada operación
  \item Lo que puede hacer un proceso depende de los contextos de seguridad
  \item Cada fichero o proceso tiene asociado un contexto de SELinux
  \item El contexto se define por la identidad del usuario que lo inicia, el rol, el tipo y el nivel de seguridad
  \end{itemize}
  \begin{description}
  \item[Identidad] Usuario de SELinux, múltiples usuarios pueden usar la misma identidad SELinux, sufijo \texttt{\_u}
  \item[Rol] Se pueden asociar diferentes roles a cada identidad según sea necesario, sufijo \texttt{\_r}
  \item[Tipo (dominio)] Asociado al tipo de proceso, sufijo \texttt{\_t}. Usado por la mayoría de políticas
  \item [Nivel de seguridad] Utilizado en entornos complejos
  \end{description}
\end{frame}
\begin{frame}[fragile]
  \frametitle{\texttt{-Z}}
Las instrucciones relevantes que necesiten información sobre los contextos incluyen el argumento \texttt{-Z}:
\begin{lstlisting}[basicstyle=\tiny\ttfamily]
id -Z
unconfined_u:unconfined_r:unconfined_t:s0-s0:c0.c1023

ps Z
LABEL                             PID TTY      STAT   TIME COMMAND
unconfined_u:unconfined_r:unconfined_t:s0-s0:c0.c1023 5253 pts/0 Ss   0:00 -bash
unconfined_u:unconfined_r:unconfined_t:s0-s0:c0.c1023 5422 pts/0 R+   0:00 ps Z

ls -lZ
total 0
-rw-rw-r--. 1 centos centos unconfined_u:object_r:user_home_t:s0 0 feb  3 10:19 borrame

sudo ls -lZ /root/
total 16
-rw-------. 1 root root system_u:object_r:admin_home_t:s0 5589 ene 13  2020 anaconda-ks.cfg
-rw-------. 1 root root system_u:object_r:admin_home_t:s0 5355 ene 13  2020 original-ks.cfg
\end{lstlisting}
\end{frame}

\begin{frame}[fragile]
  \frametitle{Aspectos a considerar}
  Cuando se interactúa con SELinux podemos encontrar diferentes situaciones:
  \begin{itemize}
  \item Utilizar los diferentes procesos dentro de lo permitido para
    cada dominio. Funcionamiento normal
  \item Utilizar algún proceso confinado fuera de los límites permitidos
    \begin{itemize}
    \item Buscamos en Internet: \texttt{\textst{setenforce 0}}
    \item Pasamos el proceso a modo permisivo:
    \begin{lstlisting}
      semanage permissive -a httpd_t
    \end{lstlisting}
    \item Modificamos el proceso y/o el contexto para que funcione correctamente
    \end{itemize}
  \item Definir o modificar la política para un proceso. Uso más avanzado
  \end{itemize}
\end{frame}

\begin{frame}[fragile]
  \frametitle{Problemas típicos con un proceso confinado}
  \begin{itemize}
  \item Ejecutamos un servicio en un puerto no permitido
    \begin{lstlisting}[basicstyle=\scriptsize\ttfamily]
semanage port -l|grep ^http_port
http_port_t                    tcp      80, 81, 443, 488, 8008, 8009, 8443, 9000
    \end{lstlisting}
  \item Un servicio lee o escribe ficheros de un tipo no definido en la política:
    \begin{lstlisting}[basicstyle=\tiny\ttfamily]
ls -Zl /var/www/html/
-rw-r--r--. 1 root root unconfined_u:object_r:samba_share_t:s0       5 Feb  4 15:56 index2.html
-rw-r--r--. 1 root root unconfined_u:object_r:httpd_sys_content_t:s0 5 Feb  4 16:03 index.html
    \end{lstlisting}
  \end{itemize}
\end{frame}

\begin{frame}[fragile]
  \frametitle{Modificar un contexto}
  \begin{description}
  \item[\texttt{chcon}] \texttt{chcon -t httpd\_sys\_content\_t index2.html}
  \item[restorecon] \texttt{restorecon index2.html}
    
    Se puede usar recursivamente (\texttt{-R})
  \item[semanage] 
    Supongamos el caso en el que pongamos DocumentRoot en
    \texttt{/srv/www}. Debemos modificar la política para que permita
    poner contenido web en ese directorio:
    \begin{lstlisting}[basicstyle=\scriptsize\ttfamily]
      semanage fcontext  -a -t httpd_sys_content_t   ``/srv/www(/.*)?''
    \end{lstlisting}
    Además debemos ejecutar \texttt{restorecon} sobre el directorio
    porque las políticas de SELinux se leen solo al iniciar el
    equipo
  \end{description}
\end{frame}

\begin{frame}[fragile]
  \frametitle{Herramientas adicionales}
  \begin{itemize}
  \item SELinux registra la actividad relevante en \texttt{/var/log/audit/audit.log}
  \item Cada registro de audit.log tiene un código de registro (\texttt{[0-9]{10}.[0-9]{3}:[0-9]{2}})
  \item Podemos pasar el código de registro como parámetro a \texttt{audit2why} o \texttt{audit2allow}:
    \begin{lstlisting}
      grep 1612454328.715:412 /var/log/audit/audit.log |audit2why
    \end{lstlisting}
  \item Instalamos el paquete \texttt{setools-console}
    \begin{lstlisting}
      sesearch -A -s httpd_t
      ...
      ls -Zl /usr/sbin/httpd
      ...
      sesearch -A -s httpd_t -t httpd_exec_t
    \end{lstlisting}
  \end{itemize}
\end{frame}
\end{document}
