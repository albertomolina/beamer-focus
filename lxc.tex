\documentclass[aspectratio=169]{beamer}
\usepackage[utf8]{inputenc}
\usepackage[T1]{fontenc}

\usetheme{focus}
\definecolor{main}{RGB}{11, 93, 25}

\usecolortheme[RGB={11,93,20}]{structure}
\usepackage[spanish]{babel}

\usepackage{colortbl}
\usepackage{color}
\usepackage{pifont}
\usepackage{ulem}



\usepackage{colortbl}
\usepackage{color}
\usepackage{pifont}
% \usepackage[normalem]{ulem}                                                                                                                                
\usepackage{hyperref}
\hypersetup{
    colorlinks=true,
    linkcolor=blue,
    filecolor=magenta,
    urlcolor=cyan,
}

\usepackage{listings}

\parskip=12pt

\lstset{basicstyle=\normalsize,
aboveskip=5pt,
basicstyle=\small\ttfamily,
belowskip=5pt
}

\author{Alberto Molina Coballes\\                                                                                                                            
José Domingo Muñoz Rodríguez}
\title{LXC}
\institute{IES Gonzalo Nazareno}
\titlegraphic{\includegraphics[width=1.5cm]{cc_by_sa.png}}
\logo{\includegraphics[width=.75cm]{logo_iesgn.png}}
\date{\today}

\definecolor{verde}{rgb}{0,0.73,0}

\begin{document}
\begin{frame}[t,plain]
\titlepage
\end{frame}

\begin{frame}
  \frametitle{Virtualización ligera o por contenedores}
  \begin{itemize}
  \item Común en sistemas Solaris o FreeBSD desde hace años
  \item Todas las máquinas virtuales utilizan el kérnel del anfitrión
  \item Linux: OpenVZ, vserver, ... Importantes limitaciones
  \item Desarrollo en Linux de espacios de nombres (\textit{namespaces}), grupos de control (cgroups), etc. permite crear sistemas de contenedores para linux: lxc, systemd-nspawn
  \end{itemize}
\end{frame}
\begin{frame}
  \frametitle{Características principales}
  \begin{itemize}
  \item Sustituye a sistemas anteriores como OpenVZ o Linux vservers
  \item Importantes mejoras al usar
    \begin{itemize}
    \item Espacios de nombres del kérnel 
    \item Apparmor y SELinux
    \item Chroots (pivot\_root)
    \item Kernel capabilities
    \item CGroups
    \end{itemize}
  \item Totalmente integrado en el kérnel linux
  \end{itemize}
\end{frame}

\begin{frame}
  \frametitle{Proyecto LXC}
  \begin{itemize}
  \item Comienza el desarrollo en 2008
  \item Licencia LGPL
  \item Desarrollado principalmente por Canonical
  \item Lenguaje C
  \item \url{http://linuxcontainers.org}
  \end{itemize}
\end{frame}

\begin{frame}[fragile]
  \frametitle{Instalación}
  No hay binarios raros, instalaciones en \texttt{/opt} o \texttt{install.sh}, simplemente:
\begin{verbatim}
apt install lxc
\end{verbatim}
\end{frame}

\begin{frame}
  \frametitle{Uso de LXC}
  \begin{itemize}
  \item Pertenece a los denominados contenedores de sistemas
  \item Gestiona contenedores directamente sin ``adornos'' y a bajo nivel
  \item Los contenedores de aplicaciones: docker, rkt, \ldots están pensados para el despliegue de aplicaciones en arquitectura de microservicios
  \item No compite con docker sino con otros sistemas de virtualización
  \item No hay nuevos conceptos, es otro sistema de virtualización en la que todos los contenedores tienen el mismo kérnel
  \item Utiliza pivot root para definir el directorio raíz del contenedor en un directorio
  \item No hay que definir un LXCFile ni nada que se parezca ;)
  \item Para acceder al contenedor utilizamos ssh(!)
  \item LXD: LXC + demonio + CLI unificado + imágenes
  \end{itemize}
\end{frame}

\begin{frame}
  \frametitle{Integración}
  \texttt{LXC} no sólo se utiliza desde línea de comandos, se puede integrar con:
  \begin{itemize}
  \item Libvirt
  \item OpenStack
  \item Opennebula
  \item Vagrant
  \item \ldots
  \end{itemize}
\end{frame}

\begin{frame}
  \frametitle{Manos a la obra}
  ¿Comenzamos? Esto se aprende haciendo
\end{frame}
\end{document}
