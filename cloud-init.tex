\documentclass{beamer}
\usepackage[utf8]{inputenc}
\usepackage[T1]{fontenc}

\usetheme{focus}
\definecolor{main}{RGB}{11, 93, 25}

\usecolortheme[RGB={11,93,20}]{structure}
\usepackage[spanish]{babel}

\usepackage{colortbl}
\usepackage{color}
\usepackage{pifont}
\usepackage{ulem}



\usepackage{colortbl}
\usepackage{color}
\usepackage{pifont}
% \usepackage[normalem]{ulem}
\usepackage{listings}

\parskip=12pt

\lstset{basicstyle=\small,
aboveskip=5pt,
basicstyle=\tiny\ttfamily,
belowskip=5pt
}

\author{Alberto Molina Coballes}
\title{\emph{cloud-init}}
\institute{IES Gonzalo Nazareno}
\titlegraphic{\includegraphics[width=1.5cm]{cc_by_sa.png}}
\logo{\includegraphics[width=.75cm]{logo_iesgn.png}}
\date{\today}

\definecolor{verde}{rgb}{0,0.73,0}

\begin{document}

\def\braces#1{[#1]}

\begin{frame}[t,plain]
\titlepage
\end{frame}

\begin{frame}
  \frametitle{Configuración de una instancia}
  \begin{itemize}
  \item La creación de un servidor en IaaS (instancia) es un proceso
    peculiar, que requiere un tratamiento específico.
  \item Se parte de una imagen mínima, sin contraseña establecida y
    con una configuración totalmente genérica.
  \item Imprescindible la configuración inicial:
    \begin{itemize}
    \item Generación de la clave ssh de la instancia
    \item Parámetros de red, hostname, etc.
    \item Autenticación del usuario (clave ssh)
    \end{itemize}
  \item Configuración no esencial
  \end{itemize}
\end{frame}

\begin{frame}
  \frametitle{El proyecto cloud-init}
  \begin{itemize}
  \item cloud-init: cloud instance initialization
  \item Estándar de facto en nube pública o privada
  \item Desarrollado en python
  \item Proyecto liderado por Canonical
  \item Paquete \texttt{cloud-init} instalado habitualmente en las
    imágenes para IaaS
  \item Documentación: \url{https://cloudinit.readthedocs.io}
  \end{itemize}
\end{frame}

\begin{frame}
  \frametitle{Datos para la instancia}
  Una instancia que se inicia o reinicia, puede obtener diferentes tipos de datos, en función del origen de estos:
  \begin{itemize}
  \item Metadatos: Obtenidos del servidor de metadatos del proveedor de nube (típicamente a través de dirección de enlace local \url{http://169.254.169.254}). Incluye las características propias de la instancia: nombre, configuración de red, tamaño de los discos, etc.
  \item Datos de usuario (opcional): Datos adicionales de configuración que proporciona el usuario de la instancia (\textit{user-data})
  \item Datos del proveedor (opcional): Datos adicionales de configuración proporcionados por el proveedor (\textit{vendor-data})
  \end{itemize}
\end{frame}

\begin{frame}
  \frametitle{Etapas}
  \begin{enumerate}
  \item Generator
  \item Local
  \item Network
  \item Config
  \item Final
  \end{enumerate}
\end{frame}

\begin{frame}
  \frametitle{Generator}
  \begin{itemize}
  \item Un generador de systemd activará cloud-init, salvo que:
    \begin{itemize}
    \item Exista el fichero \texttt{/etc/cloud/cloud-init.disabled}
    \item Se use en el arranque la opción de kérnel \texttt{cloud-init=disabled}
    \end{itemize}
  \end{itemize}
\end{frame}

\begin{frame}[fragile]
  \frametitle{Local}
  \begin{itemize}
  \item Asociada al servicio \texttt{cloud-init-local.service}
  \item Obtiene la información de la configuración de red:
  \end{itemize}
  \begin{lstlisting}
+--------+------+------------------------------+---------------+--------+--------------+
| Device |  Up  |           Address            |      Mask     | Scope  | Hw-Address   |
+--------+------+------------------------------+---------------+--------+--------------+
|  eth0  | True |          10.0.0.10           | 255.255.255.0 | global | fa:16:3e:... |
|  eth0  | True | fe80::f816:3eff:fe4f:391a/64 |       .       |  link  | fa:16:3e:... |
|   lo   | True |          127.0.0.1           |   255.0.0.0   |  host  |         .    |
|   lo   | True |           ::1/128            |       .       |  host  |         .    |
+--------+------+------------------------------+---------------+--------+--------------+
\end{lstlisting}
\begin{itemize}
\item Si no se obtiene configuración de la red, se hace una petición DHCP
\end{itemize}
\end{frame}

\begin{frame}
  \frametitle{Network}
  \begin{itemize}
  \item Asociada al servicio \texttt{cloud-init.service}
  \item Se ejecuta después de que esté configurada la red:
    \begin{itemize}
    \item Redimensión de la partición raíz
    \item Montaje
    \item Generación de números aleatorios
    \item hostname
    \item Generación de las claves ssh del servidor
    \item ...
    \end{itemize}
  \end{itemize}
\end{frame}

\begin{frame}
  \frametitle{Config}
  \begin{itemize}
  \item Asociada al servicio \texttt{cloud-config.service}
  \item Configuración ntp
  \item Configuración zona horaria
  \item Configuración de apt/dpkg ...
  \item Configuración de contraseñas, claves ssh
  \end{itemize}
\end{frame}

\begin{frame}
  \frametitle{Final}
  \begin{itemize}
  \item Asociada al servicio \texttt{cloud-final.service}
  \item Actualización de los paquetes instalados (update/upgrade)
  \item Ejecución de los datos de proveedor (\textit{vendor-data})
  \item Ejecución de scripts de usuario
  \end{itemize}
\end{frame}

\begin{frame}
  \frametitle{Configuración de cloud-init}
  \begin{itemize}
  \item La utilización de metadatos depende principalmente del proveedor de nube
  \item El usuario utiliza cloud-init principalmente de dos formas:
    \begin{itemize}
    \item Mediante \textit{user-data}
    \item Modificando el contenido \texttt{/etc/cloud/}. Más apropiado
      para la configuración de la imagen.
    \end{itemize}
  \end{itemize}
\end{frame}

\begin{frame}
  \frametitle{user-data: cloud-config}
  Hay varios formatos aceptados para introducir ``user-data'', el más
  habitual es mediante el formato YAML conocido como cloud-config.
  \begin{itemize}
  \item
    \url{https://cloudinit.readthedocs.io/en/latest/topics/examples.html}
  \item \url{https://cloudinit.readthedocs.io/en/latest/topics/modules.html}
  \end{itemize}
\end{frame}

\begin{frame}[fragile]
  \frametitle{Comprobaciones}
  Se puede usar la instrucción \texttt{cloud-init} para realizar comprobaciones:
  \begin{lstlisting}
    cloud-init query -l
    cloud-init init
    cloud-init analyze show
  \end{lstlisting}
\end{frame}
\end{document}
